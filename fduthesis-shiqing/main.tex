% \iffalse meta-comment
%
% Copyright (C) 2018--2023 by Xiangdong Zeng <xdzeng96@gmail.com>
%
% This work may be distributed and/or modified under the
% conditions of the LaTeX Project Public License, either
% version 1.3c of this license or (at your option) any later
% version. The latest version of this license is in:
%
%   http://www.latex-project.org/lppl.txt
%
% and version 1.3 or later is part of all distributions of
% LaTeX version 2005/12/01 or later.
%
% This work has the LPPL maintenance status `maintained'.
%
% The Current Maintainer of this work is Xiangdong Zeng.
%
% \fi

%*********************************************************************
% fduthesis: 复旦大学论文模板
% 2023-05-27 v0.9a
%
% 重要提示:
%   1. 请确保使用 UTF-8 编码保存
%   2. 请使用 XeLaTeX 或 LuaLaTeX 编译
%   3. 请仔细阅读用户文档
%   4. 修改、使用、发布本文档请务必遵循 LaTeX Project Public License
%   5. 不需要的注释可以尽情删除
%*********************************************************************

\documentclass[type=master]{fduthesis}
% 模板选项:
%   type = doctor|master|bachelor  论文类型,默认为本科论文
%   oneside|twoside                论文的单双面模式,默认为 twoside
%   draft = true|false             是否开启草稿模式,默认关闭
% 带选项的用法示例:
%   \documentclass[oneside]{fduthesis}
%   \documentclass[twoside, draft=true]{fduthesis}
%   \documentclass[type=bachelor, twoside, draft=true]{fduthesis}

\fdusetup{
  % 参数设置
  % 允许采用两种方式设置选项:
  %   1. style/... = ...
  %   2. style = { ... = ... }
  % 注意事项:
  %   1. 不要出现空行
  %   2. “=” 两侧的空格会被忽略
  %   3. “/” 两侧的空格不会被忽略
  %   4. 请使用英文逗号 “,” 分隔选项
  %
  % style 类用于设置论文格式
  style = {
    % font = times,
    % 西文字体(包括数学字体)
    % 允许选项:
    %   font = garamond|libertinus|lm|palatino|times|times*|none
    %
    % cjk-font = fandol,
    % 中文字体
    % 允许选项:
    %   cjk-font = adobe|fandol|founder|mac|sinotype|sourcehan|windows|none
    %
    % 注意:
    %   1. 中文字体设置高度依赖于系统。各系统建议方案:
    %        windows:cjk-font = windows
    %        mac:    cjk-font = mac
    %        linux:  cjk-font = fandol(默认值)
    %   2. 除 fandol 和 sourcehan 外,其余字体均为商用字体,请注意版权问题
    %   3. 但 fandol 字体缺字比较严重,而 sourcehan 没有配备楷体和仿宋体
    %   4. 这里中西文字体设置均注释掉了,即使用默认设置:
    %        font     = times
    %        cjk-font = fandol
    %   5. 使用 font = none / cjk-font = none 关闭默认字体设置,需手动进行配置
    %
    % font-size = -4,
    % 字号
    % 允许选项:
    %   font-size = -4|5
    %
    % fullwidth-stop = catcode,
    % 是否把全角实心句点 “.” 作为默认的句号形状
    % 允许选项:
    %   fullwidth-stop = catcode|mapping|false
    % 说明:
    %   catcode   显式的 “。” 会被替换为 “.”(e.g. 不包括用宏定义保存的 “。”)
    %   mapping   所有的 “。” 会被替换为 “.”(使用 LuaLaTeX 编译则无效)
    %   false     不进行替换
    %
    footnote-style = xits,
    % 脚注编号样式
    % 允许选项:
    %   footnote-style = plain|libertinus|libertinus*|libertinus-sans|
    %                    pifont|pifont*|pifont-sans|pifont-sans*|
    %                    xits|xits-sans|xits-sans*
    % 默认与西文字体保持一致
    %
    % hyperlink = color,
    % 超链接样式
    % 允许选项:
    %   hyperlink = border|color|none
    %
    % hyperlink-color = default,
    % 超链接颜色
    % 允许选项:
    %   hyperlink-color = default|classic|material|graylevel|prl
    %
    bib-backend = bibtex,
    % 参考文献支持方式
    % 允许选项:
    %   bib-backend = bibtex|biblatex
    %
    % bib-style = numerical,
    % 参考文献样式
    % 允许选项:
    %   bib-style = author-year|numerical|<其他样式>
    % 说明:
    %   author-year  著者—出版年制
    %   numerical    顺序编码制
    %   <其他样式>   使用其他 .bst(bibtex)或 .bbx(biblatex)格式文件
    %
    % cite-style = {},
    % 引用样式
    % 默认为空,即与参考文献样式保持一致
    % 仅适用于 biblatex;如要填写,需保证相应的 .cbx 格式文件能被调用
    %
    bib-resource = {main.bib},
    % 参考文献数据源
    % 可以是单个文件,也可以是用英文逗号 “,” 隔开的一组文件
    % 如果使用 biblatex,则必须明确给出 .bib 后缀名
    %
    % logo = {fudan-name.pdf},
    % 封面中的校名图片
    % 模版已自带,通常不需要额外配置
    %
    % logo-size = {0.5\textwidth},      % 只设置宽度
    % logo-size = {{}, 3cm},            % 只设置高度
    % logo-size = {8cm, 3cm},           % 设置宽度和高度
    % 设置校名图片的大小
    % 通常不需要调整
    %
    % declaration-page = {declaration.pdf},
    % 插入扫描版的声明页 PDF 文档
    % 默认使用预定义的声明页,但不带签名
    %
    % auto-make-cover = true
    % 是否自动生成论文封面(封一)、指导小组成员名单(封二)和声明页(封三)
    % 除非特殊需要(e.g. 不要封面),否则不建议设为 false
  },
  %
  % info 类用于录入论文信息
  info = {
    title = {基于高维特征概率密度建模的模型不确定性的研究},
    % 中文标题
    % 长标题建议使用 “\\” 命令手动换行(不是指在源文件里输入回车符,当然
    % 源文件里适当的换行可以有助于代码清晰):
    %   title = {最高人民法院、最高人民检察院关于适用\\
    %            犯罪嫌疑人、被告人逃匿、死亡案件违法所得\\
    %            没收程序若干问题的规定},
    %
    title* = {Research on Model Uncertainty Based on High-Dimensional Feature Probability Density Modeling},
    % 英文标题
    %
    author = {师清},
    % 作者姓名
    %
    % author* = {Your name},
    % 作者姓名(英文 / 拼音)
    % 目前不需要填写
    %
    supervisor = {薛向阳\quad 教授},
    % 导师
    % 姓名与职称之间可以用 \quad 打印一个空格
    %
    major = {计算机技术},
    % 专业
    %
    degree = professional,
    % 学位类型
    % 允许选项:
    %   degree = academic|professional
    % 说明:
    %   academic      学术学位
    %   professional  专业学位
    %
    department = {计算机科学与技术},
    % 院系
    %
    student-id = {22210240262},
    % 作者学号
    %
    % date = {2023 年 1 月 1 日},
    % 日期
    % 注释掉表示使用编译日期
    %
    % secret-level = ii,
    % 密级
    % 允许选项:
    %   secret-level = none|i|ii|iii
    % 说明:
    %   none  不显示密级与保密年限
    %   i     秘密
    %   ii    机密
    %   iii   绝密
    %
    % secret-year = {五年},
    % 保密年限
    % secret-level = none 时该选项无效
    %
    instructors = {
      {薛向阳 \quad 教授},
      {戈维峰     \quad 青年副研究员}
    },
    % 指导小组成员
    % 使用英文逗号 “,” 分隔
    % 如有需要,可以用 \quad 手工对齐
    %
    keywords = {模型不确定性;高斯混合模型;梯度空间;输入扰动;特征空间;辅助损失函数},
    % 中文关键词
    % 使用英文逗号 “,” 分隔
    %
    keywords* = {Model Uncertainty; Gaussian Mixed Model;Gradient Space;Input perturbation;Feature Space;Auxiliary Loss Function},
    % 英文关键词
    % 使用英文逗号 “,” 分隔
    %
    % clc = {O413.1},
    % 中图分类号
    %
    % jel = {C02},
    % JEL 分类号,仅适用于经济学院等部分院系
  }
}

% 需要的宏包可以自行调用
\usepackage{physics}
\usepackage{algorithm}
%\usepackage{algorithmic}
%\usepackage{algorithmicx}
\usepackage{algpseudocode}
\usepackage{multirow}
\usepackage{makecell} % 使用 makecell 包
\usepackage{array}    % 用于表格的灵活控制
\newcolumntype{P}[1]{>{\centering\arraybackslash}p{#1}}
% 需要的命令可以自行定义
\newcommand{\hilbertH}{\symcal{H}}
\newcommand{\ee}{\symrm{e}}
\newcommand{\ii}{\symrm{i}}

\begin{document}

% 这个命令用来关闭版心底部强制对齐,可以减少不必要的 underfull \vbox 提示,但会影响排版效果
% \raggedbottom

% 前置部分包含目录、中英文摘要以及符号表等
\frontmatter

% 目录
\tableofcontents
% 插图目录
\listoffigures
% 表格目录
\listoftables

\begin{abstract}
近年来,深度学习在许多领域都有着广泛的应用,算法的准确率也在不断提高。但是在一些安全敏感的领域,如自动驾驶、医疗图像识别,即使模型以很低的概率预测错误,仍然会给应用带来严重的后果,极大限制了深度学习算法在这些领域的应用。于是就兴起了对神经网络不确定性的研究,神经网络的不确定性表示了神经网络对本次预测结果的信心,如果神经网络不确定性高,说明本次预测可靠程度低,通过选取一个阈值,对于不确定性高于某个阈值的预测,引入人类的判断,来降低线上系统事故发生的可能性。

对神经网络的建模有不同的思路,如贝叶斯神经网络(bnn),模型集成(Ensemble)的方法。本文采用的单一确定性神经网络建模不确定性的思路,通过对神经网络抽取的高维特征进行概率密度建模,并使用对数概率密度表示神经网络预测的模型不确定性。这种思路在DDU算法\cite{Mukhoti_2023_CVPR}中已经有了很好的效果,但是实验观察到相对于ensemble类的思路,DDU算法仍有差距,于是本文提出两点改进,提高了这一建模思路的效果。

首先是基于梯度空间的分析,对输入图片加上与梯度相关的扰动。通过可视化loss关于输入图片的梯度响应,观察到域内样本和域外样本存在明显的差异,域外样本的梯度较大,域内样本的梯度较小。对输入图片加入与梯度相关的输入扰动, 再用重新计算出的对数概率密度表示模型不确定性,通过在OOD检测任务,对抗样本检测任务上的实验,观察到方法对于模型不确定性建模的提升。

第二点改进是使用TSNE对高维特征降维到二维空间,然后对高维特征进行可视化分析,在此基础上本文指出通过提高特征空间的类内紧密性和类间可分性,可以进一步提高高维特征概率密度建模不确定性的效果,为了达到提高特征空间的类内紧密性和类间可分性这一目的,本文实验了几种辅助损失函数,和交叉熵损失函数一起联合训练神经网络模型,最终提出使用FeatureLoss,

总之,本文提出两个方法,改进了高维特征概率密度建模不确定性的效果,推进了深度学习算法在某些安全敏感领域的应用。
\end{abstract}

\begin{abstract*}
In recent years, deep learning has found extensive applications across various fields, achieving continually improving accuracy in its algorithms. However, in safety-critical domains such as autonomous driving and medical image recognition, even low-probability prediction errors can result in severe consequences, significantly limiting the application of deep learning algorithms in these areas. This challenge has led to a growing interest in studying the uncertainty of neural networks. Neural network uncertainty quantifies the confidence a network has in its predictions. High uncertainty implies low reliability of a prediction, and by setting a threshold, predictions with uncertainty above this threshold can be flagged for human review, thereby reducing the risk of accidents in online systems.

There are various approaches to modeling neural network uncertainty, such as Bayesian Neural Networks (BNNs) and ensemble methods. This paper adopts an approach based on a single deterministic neural network to model uncertainty. It models the probability density of high-dimensional features extracted by the network and uses the log-probability density to represent the uncertainty of predictions. This approach has demonstrated promising results in the DDU algorithm \cite{Mukhoti_2023_CVPR}, but experiments reveal that it still lags behind ensemble-based methods. To address this gap, this paper proposes two improvements to enhance the effectiveness of this modeling approach.

The first improvement involves gradient-space analysis by introducing gradient-related perturbations to the input images. By visualizing the loss gradient responses concerning the input images, it was observed that there are significant differences between in-domain and out-of-domain samples: the gradients for out-of-domain samples are generally larger, while those for in-domain samples are smaller. By adding gradient-related perturbations to the input images and recalculating the log-probability density to represent model uncertainty, experiments on OOD detection and adversarial sample detection tasks demonstrated improved modeling of model uncertainty.

The second improvement leverages t-SNE to reduce high-dimensional features to a two-dimensional space for visualization. Based on this analysis, the paper highlights that improving intra-class compactness and inter-class separability in the feature space can further enhance the effectiveness of probability density modeling for uncertainty. To achieve this, various auxiliary loss functions were experimented with and combined with the cross-entropy loss function to jointly train the neural network model. Ultimately, the paper proposes the use of FeatureLoss.

In conclusion, this paper introduces two methods to improve the effectiveness of high-dimensional feature probability density modeling for uncertainty, advancing the application of deep learning algorithms in safety-critical domains.
\end{abstract*}

% 符号表
% 语法与 LaTeX 表格一致:列用 & 区分,行用 \\ 区分
% 如需修改格式,可以使用可选参数:
%   \begin{notation}[ll]
%     $x$ & 坐标 \\
%     $p$ & 动量
%   \end{notation}
% 可选参数与 LaTeX 标准表格的列格式说明语法一致
% 这里的 “ll” 表示两列均为自动宽度,并且左对齐
\begin{notation}[ll]
  $D$                  & 数据集        \\

\end{notation}

% 主体部分是论文的核心
\mainmatter

% 建议采用多文件编译的方式
% 比较好的做法是把每一章放进一个单独的 tex 文件里,并在这里用 \include 导入,例如
%   \include{chapter1}
%   \include{chapter2}
%   \include{chapter3}

\chapter{绪论}
\section{研究背景和意义}
近年来随着计算机硬件的快速发展和数据的爆炸性增长,深度学习算法研究取得了显著进展,在图像识别\cite{krizhevsky2012imagenet}、自然语言处理\cite{brown2020language}、自动驾驶\cite{teichmann2018multinet}、医疗诊断\cite{gulshan2016development}等领域取得了广泛的应用。现阶段的深度学习算法在准确性上已经可以达到很好的效果,但仍普遍存在过度自信问题(Overconfident issue),即深度神经网络在其预测结果上往往过于自信,即使在它不熟悉的输入上,也可能错误地给出过高的预测概率。在一些高风险的领域,比如医疗诊断,自动驾驶等,错误的预测可能会带来严重的线上事故,这使得神经网络在这些领域的应用仍然受到限制。因此,在实际应用中,不仅需要模型能准确给出预测结果,还希望模型能给出一个值指示本次预测结果可靠程度,这就是神经网络的\textbf{不确定性(Uncertainty)}。神经网络预测结果的不确定性较高,就可以在必要时引入人类监督或放弃高风险决策,规避风险和灾难性后果。

对神经网络不确定性的相关研究具有重大意义。在实际应用中,对神经网络不确定性的研究有助于提升模型的鲁棒性,帮助模型更稳健、更可靠地应用于医疗、自动驾驶、金融等具体领域。不确定性量化(Uncertainty Quantification,UQ)可以帮助模型识别出对某些输入的预测信心不足的情况,避免错误决策引发的系统事故。

神经网络不确定性的研究对神经网络的可解释性也有重要意义,通过对神经网络不确定性的研究,学术界和工业界能够更深入地理解神经网络的行为和特性。当前多数研究和应用都把神经网络作为“黑箱”模型,其缺乏可解释性和透明性的问题长期以来阻碍了其在医疗、自动驾驶等安全关键领域的应用\cite{ROY201911}\cite{christoph2020interpretable}。通过不确定性研究,模型可以对自己的预测进行自我评价,为用户提供更直观的信心指标。这不仅有助于用户更好地理解和使用神经网络系统,还能增加对人工智能技术的信任。

最后,对神经网络不确定的研究也推动了深度学习理论的发展。当前,有关神经网络不确定性的研究已经催生了许多新的方法,例如深度高斯过程、贝叶斯神经网络、蒙特卡罗随机失活(MC dropout)、对抗训练等。这些方法不仅拓展了神经网络的理论基础,也为其他机器学习问题(如模型选择、主动学习、强化学习)提供了新的解决思路。



\section{不确定性的来源与分类}
\textbf{神经网络不确定性(Uncertainty in Neural Networks)}是指在神经网络的预测中,描述其对预测结果的置信程度\cite{abdar2021review}。神经网络的不确定性量化在许多实际应用中非常重要,尤其是在需要高可靠性和安全性的领域,如自动驾驶、医疗诊断和金融预测\cite{gawlikowski2023survey}\cite{he2023survey}\cite{gal2016uncertainty}。在实际应用中,神经网络中的不确定性通常可以分为两类\cite{abdar2021review}:\textbf{模型不确定性}(Model Uncertainty)和\textbf{ 数据不确定性}(Data Uncertainty)。

\textbf{模型不确定性},又称为\textbf{认知不确定性}(Epistemic Uncertainty),表示所训练的神经网络模型本身能力的局限性,模型不确定性可能来源于模型结构,训练过程,训练数据不足。当训练数据不足时,模型的参数未能很好地捕捉数据分布分布,导致预测的结果含有很大的模型不确定性。模型不确定性可以通过改变训练模型,优化训练过程,增加训练数据来减少。如图\ref{fig:uncertainty},在训练数据以外所呈现的高不确定性是模型不确定性。\textbf{数据不确定性}(Data Uncertainty), 又称为\textbf{偶然不确定性}(Aleatoric Uncertainty),通常是由数据本身的随机性或噪声引起,可能来源于数据收集时潜在的噪声或者数据标注时的错误。数据不确定性无法通过任务方式被消除。如图\ref{fig:uncertainty},种间训练数据本身所存在的误差和噪声引起的不确定性是数据不确定性。

\begin{figure}[H]
    \centering
    \includegraphics[width=0.9\linewidth]{assets/1-1.png}
    \caption{不确定性的来源与分类\cite{abdar2021review}
}
    \label{fig:uncertainty}
\end{figure}

数据不确定性和模型不确定性并非绝对的,可能会随着问题的定义有所变化\cite{hullermeier2021aleatoric}。Valdenegro-Toro等人\cite{valdenegro2022deeper}也在研究中发现,在解耦不确定性的过程中,模型不确定性和数据不确定性会相互影响,这一点是出乎意料的,因为通常认为只有模型不确定性应该与模型交互,而数据不确定性则不应和模型有关。



\section{不确定性研究的应用}
神经网络的不确定性研究(Uncertainty in Neural Networks)在许多应用场景中都具有重要价值,特别是在需要对预测结果的可靠性、稳定性和信心进行量化的任务中。传统的神经网络通常给出一个确定的预测结果(例如,分类标签或回归值),但并没有提供关于该结果的不确定性信息。而不确定性建模则能够量化模型对其预测的信心程度,这在多个领域中具有广泛的应用,以下简单介绍不确定性研究在自动驾驶、医疗诊断、金融、自然语言处理等领域的应用。

\textbf{1. 自动驾驶}

在自动驾驶系统中,车辆必须能够基于摄像头图像、雷达、激光雷达等数据预测和识别路况,并做出实时决策。然而,这些数据本身可能存在噪声,或者在复杂环境中不完全可靠。通过不确定性建模,自动驾驶系统可以识别哪些预测是可靠的,哪些可能需要进一步验证或修正。决策制定时结合不确定性,可以优先选择对结果不确定性较低的路径规划或行动,对不确定性较高的引入人工干预。近年来关于神经网络不确定性和自动驾驶领域的结合出现很多研究,Di Feng, Lars Rosenbaum等人\cite{feng2018towards}针对激光雷达(LiDAR)的3D车辆检测任务,提出了一种结合贝叶斯深度学习的方法,能够同时估计数据不确定性和模型不确定性,通过不确定性信息优化预测结果,增强了自动驾驶系统在复杂环境中的安全性和稳健性。Hujie Pan等人\cite{pan2020towards}针对激光雷达(LiDAR)的3D车辆检测任务,结合不确定性估计方法,对回归的坐标引入拉普拉斯分布,设计了一种改进的损失函数,获得了更具有解释性和可靠性的预测结果。SalsaNext\cite{cortinhal2020salsanext} 是一种针对自动驾驶领域设计的高效、不确定性感知的语义分割模型,用于处理 LiDAR 点云数据。它在原有 SalsaNet\cite{Aksoy2019SalsaNetFR}的基础上进行了改进,结合了注意力机制和贝叶斯学习框架,实现了更高的精度和不确定性估计能力,为自动驾驶中的环境感知提供了可靠的解决方案。赵洋\cite{QCYK202405002}深入研究了自动驾驶领域目标检测任务上的不确定性估计的方法,利用不确定性信息来提高目标检测的准确性,并总结了不确定性估计的评价指标。

\textbf{2. 医疗诊断}

在智能辅助诊断领域,尤其是在面对复杂或模糊的医学影像时,对不确定的研究能够为医生提供关于某项影像诊断的置信度,帮助医生判断是否需要进一步的检查或其他医学手段。在疾病预测方向上,基于患者历史数据和医学影像,模型预测某种疾病的风险,并通过不确定性量化本次预测的可信度,对不确定度高的预测引入专业医生的判断,可以减少误诊率,降低医疗事故的可能性。在医学图像检测或者分割任务中,某些图像可能很模糊,此时模型不仅需要做出准确的预测,还需要能够提供关于其预测结果不确定性的估计,通过不确定性估计,模型可以指出分割结果的可靠区域,辅助医生进行更准确的解读。不确定性在医疗诊断领域应用上的研究,Guotai Wang等人\cite{wang2019aleatoric}研究如何结合测试时增强的方法,对卷积神经网络进行不确定性估计,从而提高医学图像分割的鲁棒性。Tanya Nair等人\cite{nair2020exploring}针对病变检测和分割,提出了一种改进的U-Net架构,结合了四种不确定性度量:预测方差、蒙特卡罗样本方差、预测熵和互信息,辅助识别模型在预测中的不确定区域,提高病变分割的准确性。邹可\cite{1024642336.nh}使用证据网络进行不确定性估计方法的研究,将不确定性估计应用到医学分割领域,提高了医学图像分割的准确性和可靠性。


\textbf{3. 金融风控与预测}

在金融行业中,不确定性量化有助于提高风险管理和预测系统的可靠性,尤其是在面对市场波动和不确定环境时。在信贷风险评估上,通过不确定性量化,金融机构能够识别贷款申请中哪些因素对违约预测有较高不确定性,从而采取更谨慎的风险管理措施。在市场预测上,量化市场趋势预测中的不确定性,可以帮助投资者做出更明智的决策,尤其在市场波动较大的情况下。在金融行业中,关于不确定性的研究,WONG SY等人\cite{wong2025quantifying}研究了不确定性量化对复杂金融时间序列预测的必要性,聚焦于时间序列中的波动性聚集特性(如资产回报率预测),提出了一种结合模型集成和证据网络的方法,显著提高了金融市场中加密货币和股票的预测的不确定性量化效果。

\textbf{4. 自然语言处理}

在自然语言处理任务中,神经网络通常需要处理复杂的文本和语言,不仅要生成准确的输出,还要提供预测的不确定性。例如,在机器翻译、情感分析或问答系统中,了解模型对预测结果的置信度对于理解其输出的可信度非常重要。在机器翻译领域,模型可以提供翻译的置信水平,帮助用户理解哪些翻译结果可靠,哪些可能需要进一步验证。在情感分析领域,评估情感分析模型的预测可信度,帮助商家了解用户评论中的潜在情感。在自动问答系统中,不确定性建模可以提供关于某个答案的置信度,帮助系统在遇到不确定的答案时请求进一步的信息或向用户确认。在自然语言处理领域关于不确定性的研究,郭晓亭\cite{1022812574.nh} 设计了一种不确定性感知的损失函数,提高了文本情感分析准确性。X. Han等人\cite{han2019attention} 提出一个基于注意力机制的神经网络框架,识别社交媒体文本中的不确定性。近年来大语言模型的兴起和广泛应用,也催生了关于不确定性在大模型方向上的研究。Lorenz Kuhn等人\cite{kuhn2023semantic}介绍了一种用于衡量大语言模型不确定性的方法,对于像问答任务这样的应用,作者提出了“语义熵”概念,通过广泛的消融实验,作者证明了语义熵在问答数据集上的表现优于其他对比基准,能够更有效地预测模型的准确性。Zhen Lin\cite{lin2024generating}针对大型语言模型(LLMs)在自然语言生成任务,特别是在黑箱模型的场景下,提出了多种不确定性量化指标,筛选不可靠的结果或将其交由人工进一步评估。

% 5. 机器人与自主系统

% 深度学习在机器人领域面临独特挑战,例如测试条件与训练分布不一致可能导致性能下降。因此,研究如何量化神经网络预测的不确定性,以避免灾难性错误,对于推动机器人技术的发展尤为重要。在路径规划与控制方面,通过不确定性量化,机器人可以识别哪些路径或动作是更安全的,哪些决策可能会导致更高的风险。在环境感知方向上,在复杂环境中(例如多变的室内环境或户外环境),机器人可以基于不确定性评估来选择更可靠的传感器数据进行决策。有关不确定性量化在机器人方向上的研究,Peretroukhin, Valentin and Giamou\cite{peretroukhin2020smooth}引入了一种概率分布表示,将旋转表示和不确定性建模相结合,可以更准确地学习旋转,同时提供对模型置信度的估计,从而在姿态估计和物体跟踪任务中表现更出色。。Yang等人\cite{yang2020d3vo}的研究通过联合学习场景深度、相机位姿以及不确定性估计,构建了一个端到端的框架,通过引入不确定性评估以增强对环境变化的适应性。


\section{问题定义和研究内容}
目前,关于模型不确定性的研究已经提出了多种方法,常见的技术包括贝叶斯神经网络、模型集成以及单一确定性网络建模等。每种方法在计算复杂性、存储开销和不确定性建模效果方面各不相同。贝叶斯神经网络通过为神经网络的权重分配概率分布,从而在推理过程中显式地建模模型的不确定性。其优点在于能够精确地建模模型的不确定性,并具有坚实的理论基础;缺点是训练过程较为缓慢,且计算和存储开销较大。模型集成方法通过训练多个模型并将其预测结果进行结合,从而获得预测的不确定性。集成方法优点是能够通过多个模型的结合,更好地捕捉数据的多样性,从而增强不确定性的建模能力;缺点是计算和存储成本较高,随着集成模型数量的增加,计算复杂度将迅速上升。

与贝叶斯神经网络和模型集成方法相比,单一确定性网络建模方法具有较低的计算成本。在这种方法中神经网络采用确定性架构,权重为固定值,不进行概率分布建模。为了量化不确定性,单一确定性建模方法可以通过对高维特征的概率密度估计来计算不确定性。具体而言,网络可以在训练集上学习到的高维特征进行概率分布估计,而在测试时,输入数据提取的特征将与该分布对比,从而量化预测的不确定性。相较于贝叶斯神经网络和模型集成方法,单一确定性建模方法的优点是计算和存储成本较低,避免了多次推理或训练多个模型的开销,且训练过程较为简便,然而,该方法在不确定性建模的效果上仍存在一定不足。

为更高效地建模神经网络的不确定性,本文进一步研究基于高维特征概率密度估计的模型不确定性量化方法。该方法在DDU算法\cite{Mukhoti_2023_CVPR}中已被提出并进行初步研究。尽管该算法在不确定性建模方面取得了一定进展,实验结果表明,与当前最优的模型集成方法相比,DDU算法仍存在一定差距。因此,本文的核心研究问题是:为了更高效准确地建模神经网络的不确定性,在DDU算法的基础上,进一步探索并改进基于高维特征概率密度估计的不确定性建模方法,以提升不确定性建模效果,并在域外样本检测、对抗样本识别和主动学习等任务中验证所提出改进方法的有效性。

本文的研究内容主要包括两个方面。其一,本文研究了对一个已经训练好的网络,在不重新改变训练方式的前提下,如何提高基于高维特征概率密度估计的模型不确定性的建模效果。本文通过计算输出对输入图片的梯度并绘制梯度响应图,实验观察到分布外样本和分布内训练样本的梯度分布存在显著差异,分布外样本的梯度响应显著更大。基于这一发现,本文提出将梯度相关的信息作为扰动引入到输入图片中,并重新计算其概率密度。同时,通过推导和理论分析,证明了加入梯度相关扰动能够有效拉大分布内样本和分布外样本的分布差距,从而提升不确定性建模的效果。通过在OOD检测、对抗样本识别和主动学习等任务上的实验评估,结果表明加入输入扰动可以显著提升基于高维特征概率密度建模的不确定性效果。

其二,本文研究如何通过改进训练方式,提高基于高维特征概率密度估计的模型不确定性建模效果。本文通过对神经网络提取的特征空间进行分析并指出,提升特征空间的类内紧密性和类间可分性能够显著改善基于高维特征概率密度估计的不确定性建模效果。基于度量学习的思想,本文设计了一种全新的辅助损失函数,并将其与交叉熵损失函数结合进行联合训练。在联合训练后的特征空间上建模概率分布,并使用概率密度量化模型不确定性。通过对联合训练后的特征空间进行定量分析,结果表明联合训练显著提升了类内紧密性和类间可分性。在OOD检测、误分类样本识别等任务中的评估结果进一步验证了联合训练方法能够有效提升基于高维特征概率密度的不确定性建模性能。此外,为解决高维特征导致的存储复杂度问题,本文还提出使用主成分分析对高维特征进行降维。在降维后的特征空间上进行概率分布建模不仅能够显著降低存储需求,同时还提高了建模效果。理论与实验均证明了这一策略的有效性。


本文的研究内容与创新点总结如下:
\begin{itemize}
    \item 提出概率密度关于输入的梯度在分布内样本和分布外样本之间存在显著差异,并通过推导和理论分析进行解释。
    \item 提出针对输入图片添加梯度相关扰动的方法,能够拉大分布内样本和分布外样本的分布差距,从而提升不确定性建模效果。
    \item 基于度量学习思想,设计一种新的辅助损失函数,与交叉熵损失函数联合训练,以优化特征空间的类内紧密性和类间可分性。
    \item 使用主成分分析对高维特征降维,显著降低存储复杂度,提升建模效率与效果。
\end{itemize}



\section{章节安排}
本文共分五章内容组织。

第一章主要是对不确定性的介绍,介绍了不确定性研究的背景和意义,不确定性的定义、来源与分类,不确定性研究的相关应用。

第二章介绍了关于不确定性的一些理论知识。首先介绍当前主流的建模不确定性的思路,主要包括贝叶斯神经网络,模型集成,测试数据增强,单一确定性网络建模的方法,并介绍了这些方法是如何计算不确定性的。由于在实际场景中缺乏不确定的标注,所以直接评估不确定性建模效果是很困难的,在第二章最后介绍是如何通过OOD检测、误分类样本识别、主动学习等代理任务间接评估不确定性的建模效果。

第三章主要介绍了基于输入扰动的改进策略,首先基于对梯度空间的分析,本章指出分布外样本和分布内样本的梯度分布存在显著差异,分布外样本的梯度响应显著更大,之后通过理论分析证明这一结论。基于此,在这一章提出将梯度相关的信息作为扰动引入到输入图片中,加入梯度相关扰动能够有效拉大分布内样本和分布外样本的分布差距,最后在OOD检测、误分类样本识别等代理任务上验证了改进后的方法对不确定性的建模效果的提升。

第四章主要介绍了使用辅助损失函数联合训练的改进,首先是对特征空间的分析,指出提升特征空间的类内紧密性和类间可分性能够显著改善高维特征概率密度建模的不确定性效果,进而提出一个新的辅助损失函数和交叉熵损失函数一起联合训练,为了减少模型存储还提出了对高维特征使用主成分分析降维,最后在OOD检测等任务上评估了改进后的方法的建模效果。

第五章是对本文研究的总结和对神经网络不确定性研究的未来的展望。

\chapter{相关理论与工作}
\section{引言}
本章从神经网络不确定性的建模方法、不确定性的计算,以及不确定性建模方法的评估三个方面,系统介绍神经网络不确定性的相关理论和研究工作。近年来,针对神经网络不确定性的建模方法层出不穷,包括基于单个神经网络的建模、集成多模型的建模,以及基于贝叶斯方法的神经网络建模,这些方法为不确定性的量化提供了多样化的理论基础和实现路径。

在不确定性的计算方面,研究主要集中于如何生成一个标量来量化神经网络的不确定性。部分方法将不确定性细分为数据不确定性和模型不确定性,分别进行建模;而另一些方法则直接对整体不确定性进行计算。

对于不确定性建模方法的评估,由于大多数情况下缺乏明确的不确定性标签,难以直接衡量方法的优劣。因此,通常通过代理任务(如分布外样本检测、对抗样本识别等)的表现,间接反映建模方法的有效性和性能。

以下将从上述三个方面逐一展开论述。


\section{不确定性的建模方法}
目前,在建模神经网络不确定性的方向上,相关研究提出了许多不同的方法:
\begin{itemize}
    \item 贝叶斯神经网络:通过在神经网络中引入贝叶斯推断,建模权重参数的不确定性,从而获得模型的输出不确定性。
    \item 模型集成的方法:通过训练多个神经网络,对于同一个输入,可以得到多个预测结果,进而通过计算多个预测值的方差或者熵来建模不确定性
    \item 测试时增强的方法:通过对输入做数据增强,得到多个增强后的输入,然后进入训练好的神经网络中预测多次得到多个预测结果,进而计算模型输出的不确定性。
    \item 单一确定的神经网络来建模不确定性:这类方法使用单一确定性的神经网络直接建模不确定性。
\end{itemize}

\subsection{贝叶斯神经网络}
贝叶斯神经网络\cite{goan2020bayesian}\cite{mackay1996bayesian}\cite{jospin2022hands}是一种将贝叶斯统计方法与深度学习相结合的模型,通过对神经网络的参数引入概率分布,量化模型的不确定性。 在传统神经网络中,参数(权重和偏置)是固定值;而在贝叶斯神经网络中,如图\ref{fig:bnn},参数被视为随机变量,遵循某种概率分布(如高斯分布)。这种方法使模型输出不仅仅是一个确定值,而是一个分布,从而反映模型的置信程度。贝叶斯神经网络建模神经网络不确定性的基本思路如下:

\begin{figure}[H]
    \centering
    \includegraphics[width=0.75\linewidth]{assets/2-1.png}
    \caption{贝叶斯神经网络\cite{blundell2015weight}
}
    \label{fig:bnn}
\end{figure}


\begin{itemize}
    \item \textbf{贝叶斯推断}:
    通过贝叶斯公式来求解参数分布:
    \[
    P(\theta \mid \mathcal{D}) = \frac{P(\mathcal{D} \mid \theta) P(\theta)}{P(\mathcal{D})}
    \]
    其中:\( P(\theta \mid \mathcal{D}) \)是后验分布,\( P(\mathcal{D} \mid \theta) \)是似然函数,\( P(\theta) \)是先验分布, \( P(\mathcal{D}) \)是边际似然,用于归一化。

    
    \item \textbf{输出分布}:
    当输入图片到贝叶斯神经网络里预测时,输出的不是单一值,而是一个分布。从关于参数的后验分\( P(\theta \mid \mathcal{D}) \)采样多个 \( \theta \),通过这些 \( \theta \) 计算多个 \( y \),对 \( y \) 进行汇总(如计算熵或方差)得到不确定性。
\end{itemize}

贝叶斯神经网络的难题是后验分布 \( P(\theta \mid \mathcal{D}) \) 的估计。由于直接计算后验分布 \( P(\theta \mid \mathcal{D}) \) 通常是不可行的,贝叶斯神经网络采用近似方法,通常的近似手段有采样方法(如马尔科夫链蒙特卡罗方法)、变分推断、拉普拉斯近似等几种。马尔科夫链蒙特卡罗(MCMC)方法是一种通过随机抽样来逼近后验分布的方法。常见的 MCMC 算法包括 Metropolis-Hastings 算法和 Gibbs 采样。它们通过生成一系列样本来估计后验分布,但计算开销较大,尤其是在高维参数空间中。变分推断通过将后验分布近似为一个简单的分布族(例如高斯分布)来进行推断。它通过最小化某种距离度量(如 Kullback-Leibler 散度)来优化参数,使得近似分布尽可能接近真实的后验分布。拉普拉斯近似通过在后验分布的最大后验估计(MAP)附近用二次泰勒展开来近似后验分布。该方法对于参数空间较小或者后验分布接近高斯分布的情况特别有效。在使用贝叶斯神经网络建模不确定性的方法中,Charles Blundell\cite{blundell2015weight}等人提出了Bayes By Backprob算法高效地计算权重参数的后验分布。Yarin Gal\cite{gal2016dropout}从理论上证明Dropout可以用作深度高斯过程上贝叶斯推断的近似,进而提出MC Dropout量化模型的不确定性。这种方法在预测阶段保留 Dropout,并通过多次前向传播获得样本分布,该方法由于模型训练和预测都比较简单,成为广为采用的估计不确定性的思路。 Hippolyt Ritter\cite{ritter2018scalable}等人通过构建Kronecker分解的拉普拉斯近似,近似求解权重参数的后验分布。Welling等人\cite{welling2011bayesian}提出了一种称为 Stochastic Gradient Langevin Dynamics (SGLD) 的方法,结合了随机梯度下降 (SGD) 和 Langevin 动力学的思想,用随机梯度近似后验分布,同时引入噪声来模仿MCMC采样过程。



\subsection{模型集成方法}
在机器学习领域,集成(Ensemble)方法\cite{lakshminarayanan2017simple}\cite{fort2019deep}是一种通过训练多个模型并集成预测来提高性能和鲁棒性的技术。在建模不确定性时,集成方法的核心思想是通过多个独立模型的多样性来量化预测结果的可信度。集成方法通过训练多个独立的模型,每个模型从不同的角度对数据进行学习,并将这些模型的预测结果进行组合(例如取平均),可以提高整体的预测性能并提供不确定性估计。以下是模型集成方法的建模步骤:


\begin{enumerate}
    \item \textbf{训练多个模型}:
    训练 \( M \) 个独立的神经网络模型 \( f_1(x), f_2(x), \ldots, f_M(x) \),其中每个模型的参数初始化不同,或者使用不同的训练数据子集。

    \item \textbf{预测分布}:
    对于给定输入 \( x \),每个模型生成一个预测 \( y_i = f_i(x) \)。集成方法的预测分布可以通过这些独立模型的输出构造, 通过汇总预测分布的统计量(如方差、熵)来评估不确定性。
\end{enumerate}

模型集成方法建模不确定性理论比较简单且具有很好的不确定性建模效果,但是需要训练多个模型,计算和存储的复杂度比较高,如何减少模型集成方法存储和计算的复杂度是这一方法的难点。Y Wen\cite{wen2020batchensemble}等人提出BatchEnsemble解决模型集成方法需要训练多个模型的存储和计算复杂度,BatchEnsemble 将一个大模型分解为多个小模型,每个小模型具有自己的参数化方式,但它们共享同一个主干模型的参数。通过共享基础权重,BatchEnsemble 避免了传统模型集成方法中每个模型都需要独立存储全部参数的缺点。BatchEnsemble 大大减少了内存占用和计算需求。




\subsection{测试时数据增强的方法}
测试时数据增强(Test Time Agmentation,TTA)的方法是一种通过对测试数据应用多种数据增强方式来估计模型预测不确定性的方法。该方法通过将同一输入的不同增强版本的预测结果进行汇总,量化模型对该输入的不确定性,该方法的建模思路如下:


\begin{enumerate}
    \item \textbf{数据增强}:
    对测试数据 \( x \),应用一系列数据增强操作 \( T_1, T_2, \ldots, T_N \),生成多个增强版本 \( \{ T_i(x) \}_{i=1}^N \)。

    \item \textbf{预测分布}:
    对每个增强版本,模型 \( f \) 生成对应的预测结果 \( \{ f(T_i(x)) \}_{i=1}^N \)。通过汇总预测分布的统计量(如方差、熵)来评估不确定性。
\end{enumerate}

TTA建模不确定性只需要单个神经网络模型且很容易实现,但是存在的一个重要的问题是如何选择有效的数据增强方式。Divya Shanmugam等人\cite{shanmugam2020and}研究了如何选择合适的增强策略,以及在结果聚合(如简单平均、加权平均)中的设计方法。




\subsection{单一确定性神经网络建模}
由于传统的建模不确定性方法计算复杂度比较高,使用单一确定性的网络一次前向计算建模不确定性的方法是现阶段的研究热点,这也是本文研究的主要内容。单一确定神经网络建模不确定性,单一是指只需要训练一个神经网络,不像模型集成方法需要训练多个网络;确定是指网络参数是确定的,不像贝叶斯神经网络一样参数符合某个分布。这类方法建模不确定性通常直接从单一确定性模型的输出中直接推导出不确定性。这类方法通过模型结构或损失函数设计,使模型能够在不需要多模型或参数采样的情况下提供对不确定性的量化,从而降低计算和存储成本。单一确定神经网络可以分为内部性方法(Internal Methods)\cite{oala2020interval}和外部性方法(Externel Methods)\cite{gawlikowski2023survey}。

内部性方法是指不使用额外的头去预测不确定性,不确定性的计算和原有神经网络的预测是一起的。Internal Methods中,最先作为不确定性研究领域基线方法的是使用最大预测概率\cite{hendrycks2017a}(Maximum Softmax Probability, MSP)作为不确定性的度量。作者观察到,OOD 样本的MSP值通常会低于inD样本,因此可以使用MSP来建模不确定性。MSP简单易实现,直接利用现有神经网络的 Softmax 输出,无需修改网络架构或增加额外的训练开销,计算仅需一次前向传播,非常适合实时场景。Prior Network\cite{malinin2018predictive} 和DBU\cite{kopetzki2021evaluating}通过引入一个显式的先验分布来对分类问题中的不确定性建模。这种方法使用一个神经网络直接对分类器的分布进行建模,而不是仅输出点估计。Prior Network 的输出是一个分类概率分布的参数,例如 Dirichlet 分布的参数$\alpha$,用以表征预测的不确定性。Murat Sensoy等人\cite{sensoy2018evidential}提出了一个新方法,通过引入主观逻辑,在神经网络的分类概率上使用狄利克雷分布来量化不确定性。该方法将神经网络的分类结果视为主观意见,网络学习如何根据数据收集证据来支持每个类别的预测。这种方法不仅能预测分类结果,还能量化预测的不确定性,从而提升模型的鲁棒性。深度核学习(Deep Kernel Learning)\cite{van2021feature}结合深度神经网络和高斯过程,将神经网络的输出作为核函数的输入,捕捉数据的不确定性。

DUQ\cite{van2020uncertainty}通过使用与径向基函数(RBF)的结构,训练一个可以在推理时通过一次前向传播计算不确定性的模型。SNGP\cite{liu2020simple}核心思想是引入输入距离感知,即模型能够量化测试样本与训练数据之间的距离,从而准确评估预测的不确定性。论文提出了谱归一化神经高斯过程(SNGP),通过在训练中加入谱归一化\cite{yoshida2017spectral}步骤和替换倒数第二层的激活函数来增强模型的距离感知能力。DDU算法\cite{Mukhoti_2023_CVPR}通过高斯混合模型在神经网络提取的高维特征上构建一个高维概率分布,同时在训练时加上谱归一化,然后使用测试样本的高维特征在该分布上的概率密度建模神经网络的不确定性,DDU算法是本文研究对比的主要方法。

外部性方法显式地在原有模型基础上额外添加一个头来预测不确定性。Maithra Raghu等人\cite{raghu2019direct}提出了一种直接不确定性预测(Direct Uncertainty Prediction, DUP)方法,通过神经网络直接预测医学诊断任务中的预测置信度。由于不确定性标注的困难性,关于这类建模不确定性方法研究很少。




\subsection{以上主流方法的对比和分析}
下面对比以上列举的四类建模不确定性方法\ref{fig:uq}的优缺点。

贝叶斯神经网络(BNN)基于贝叶斯统计理论,具备良好的可解释性和数学基础,同时通过对网络参数进行概率建模,天然具有正则化效果。此外,在数据有限的情况下,先验分布能够提供有用的信息,从而提升模型性能。然而,BNN也面临诸多挑战,包括参数分布推断和采样的高计算复杂度、网络规模扩大后推断过程的难度增加,以及先验分布选择对模型性能的敏感性。

基于集成方法的不确定性建模通过组合多个模型的预测结果,有效降低了过拟合风险,提升了预测性能。无需修改模型结构,其方法能够灵活地与多种神经网络架构融合,并在多个代理任务中表现出卓越的不确定性建模能力。然而,集成方法的劣势也较为显著。训练多个模型需要大量计算资源和存储空间,尤其在深度学习场景下,开销尤为明显。此外,集成效果依赖子模型之间的多样性和质量,若模型过于相似,集成优势可能受到削弱。  

测试时数据增强(TTA)方法在不确定性建模中具有实现简便的特点。无需对模型结构或训练过程进行修改,仅通过测试阶段的增强技术即可实现建模。然而,TTA方法需要对同一输入进行多次推理,导致推理时间增加。此外,不同增强策略对不确定性估计的效果可能存在显著差异,因此策略设计需精心考量,以确保其有效性。  

\begin{figure}[H]
    \centering
    \includegraphics[width=0.9\linewidth]{assets/uq.png}
    \caption{主流的不确定性建模方法\cite{gawlikowski2023survey}
}
    \label{fig:uq}
\end{figure}

单一确定性网络则以高效性见长。通过设计合理的输出和损失函数即可实现不确定性建模,其推理速度快、计算成本低,且易于与现有神经网络框架结合。然而,相较于集成方法,单一确定性网络在建模效果上存在一定局限性,尤其当输出分布假设未能充分捕捉数据复杂分布时,可能导致不确定性估计不够精确。  

综上所述,当前主流不确定性建模方法通常依赖多次推理或模型集成,尽管效果出色,但计算复杂度较高。单一确定性网络提供了高效的替代方案,通过减少模型复杂性和推理开销,具备更大的应用潜力。本文旨在克服现有方法的局限性,基于高维特征的概率密度进行模型不确定性评估,并提出改进策略,以提升单一确定性网络的不确定性建模能力。

\section{不确定性的计算方法}
针对不同的建模不确定性的方法,对不确定性的计算方法各不相同,在分类任务上和回归任务上,对于不确定性的计算也不相同。以下列举几种常见情况下不确定性的计算公式。

\subsection{分类任务中不确定性的计算}

对于分类任务,模型输出每个类别的预测概率,模型的不确定性可以通过评估这些预测概率的分布来衡量,常见的计算方法包括最大概率\cite{hendrycks2017a}、熵\cite{wimmer2023quantifying}、互信息\cite{wimmer2023quantifying}等。

\textbf{最大概率}
    \[
    P_{\text{max}} = \max_{i} P(y_i | \mathbf{x})
    \]
    其中,\( P(y_i | \mathbf{x}) \) 是模型对样本 \( \mathbf{x} \) 的类别 \( y_i \) 的预测概率。最大概率 \( P_{\text{max}} \) 越接近1,表示模型的确定性越高;越接近0.5(对于二分类任务),表示不确定性较高。


\textbf{熵(Entropy)}:熵是衡量概率分布不确定性的标准方法。对于分类任务,熵的计算公式为:
\[
H(\mathbf{x}) = - \sum_{i=1}^{C} P(y_i | \mathbf{x}) \log P(y_i | \mathbf{x})
\]
其中,\( C \) 是类别数,\( P(y_i | \mathbf{x}) \) 是样本 \( \mathbf{x} \) 被预测为类别 \( y_i \) 的概率。熵越大,表示模型对样本的分类越不确定。


\textbf{互信息(Mutual Information)}:互信息衡量了输入特征与输出类别之间的信息共享量。在分类任务中,输入特征 \( \mathbf{x} \) 和输出类别 \( y \) 之间的互信息可以计算为:
\[
I(\mathbf{x}; y) = H(y) - H(y | \mathbf{x})
\]
其中,\( H(y) \) 是类别 \( y \) 的熵,表示输出的总体不确定性,\( H(y | \mathbf{x}) \) 是在给定输入 \( \mathbf{x} \) 的条件下,类别 \( y \) 的条件熵,表示输入特征给出的信息后,输出类别的剩余不确定性。互信息越大,表示输入特征与输出类别之间的关系越强,从而模型的预测不确定性越小。

\subsection{回归任务中不确定性的计算}
回归任务中,使用方差\cite{kendall2017uncertainties}、区间长度\cite{pearce2018high}等计算不确定性。

\textbf{方差}:对于回归任务,预测方差表示模型预测的不确定性。假设有多个模型的预测结果或多个预测实例,可以计算预测的方差:
\[
\text{Var}(\hat{y} | \mathbf{x}) = \frac{1}{T} \sum_{t=1}^{T} \left( y^{(t)} - \bar{y} \right)^2
\]
其中,\( \bar{y} = \frac{1}{T} \sum_{t=1}^{T} y^{(t)} \) 是预测值的均值,\( T \) 是预测次数。方差越大,模型预测的不确定性越高。

\textbf{区间长度}:预测区间表示模型预测值的上下限:
\[
[\hat{y} - \delta, \hat{y} + \delta]
\]
其中,\( \delta \) 是不确定性的度量(如标准差)。较大的 \( \delta \) 表示较高的不确定性。



\section{不确定性建模的评估}
近年来,随着不确定性建模方法的不断涌现,对这些方法的评估也显得尤为重要。由于大多数不确定性任务难以直接标注不确定性(即无法获得不确定性的标签),研究中普遍采用间接方式在代理任务上进行评估。常见的代理任务包括OOD 检测、误分类检测、对抗样本检测 和 主动学习。以下对每个任务如何评估不确定性进行具体介绍。

\subsection{OOD 检测}

OOD检测(Out-of-Distribution Detection) 任务的目标是在测试过程中识别出那些来自训练集分布外的样本。OOD 样本通常与训练数据的分布显著不同。理想情况下,好的不确定性建模方法应该对 OOD 样本表现出高不确定性。在使用OOD检测任务评估不确定性建模方法的过程中,使用来自训练分布的数据集作为正样本(In-Distribution, InD),来自训练集分布外的数据集的样本作为 OOD 样本,对OOD样本和InD样本分别计算不确定性,如果计算出来的不确定性对于OOD样本和InD样本区分性比较好,那么这种不确定性建模方法就是更好的。从这个角度上来看,不确定性建模的好坏和OOD检测任务是统一的。为了度量计算出的不确定性在区分OOD样本和InD样本的好坏,通常使用机器学习中AUROC,AUPRC等指标。

AUROC(Area Under the Receiver Operating Characteristic Curve,接收者操作特征曲线下面积),是一种衡量二分类模型性能的指标。在二分类任务中,混淆矩阵(Confusion Matrix)是评估分类模型性能的一个重要工具,其定义如表\ref{confusion}:
\begin{table}[H]
\centering
\begin{tabular}{|c|c|c|}
\hline
 & \textbf{预测为正类}  & \textbf{预测为负类}  \\
\hline
\textbf{实际为正类}  & TP (True Positive) & FN (False Negative) \\
\hline
\textbf{实际为负类} & FP (False Positive) & TN (True Negative) \\
\hline
\end{tabular}
\caption{二分类任务中的混淆矩阵 (Confusion Matrix)}
\label{confusion}
\end{table}

根据以上混淆矩阵,真阳性率 (True Positive Rate, TPR)和假阳性率 (False Positive Rate, FPR)计算公式如下:
\[
\text{TPR} = \frac{\text{TP}}{\text{TP} + \text{FN}}
\]

\[
\text{FPR} = \frac{\text{FP}}{\text{FP} + \text{TN}}
\]

ROC 曲线是通过改变分类阈值来绘制的,横轴为 FPR,纵轴为 TPR。AUROC 是 ROC 曲线下的面积,当 \(\text{AUROC} = 0.5\) 时,模型没有任何区分能力,表现与随机猜测相同。当 \(\text{AUROC} = 1.0\) 时,模型能够完美区分正负类。当 \(0.5 < \text{AUROC} < 1.0\) 时,模型具有一定的区分能力,值越大表示性能越好。AUROC 是一个不依赖于特定分类阈值的评估指标,尤其适合于不平衡数据集的情况。

AUPRC(Area Under the Precision-Recall Curve,精确率-召回率曲线下面积)是评估分类模型性能的另一个常用指标,表示精度-召回率曲线下的面积。根据混淆矩阵,精度 (Precision)和召回率 (Recall)计算公式如下:

\[
\text{Precision} = \frac{\text{TP}}{\text{TP} + \text{FP}}
\]


\[
\text{Recall} = \frac{\text{TP}}{\text{TP} + \text{FN}}
\]



Precision-Recall曲线展示了不同分类阈值下的精度与召回率之间的关系。AUPRC 是该曲线下的面积。当 \(\text{AUPRC} = 0\) 时,模型完全没有区分能力。当 \(\text{AUPRC} = 1.0\) 时,模型在所有召回率下的精度都为 1,表示模型完美区分正负类。
当 \(0 < \text{AUPRC} < 1.0\) 时,模型具有一定的区分能力,AUPRC 值越大表示性能越好。AUPRC 尤其适用于不平衡数据集的情况,因为它直接衡量了在正类样本较少时的分类性能。




\subsection{误分类样本识别}


误分类样本识别的目标是利用对预测结果计算出的不确定性信息识别模型预测错误的样本。理想情况下,模型应对误分类样本表现出高不确定性。使用误分类检测作为代理任务,通过分析模型在测试数据上的误分类样本,观察不确定性分布是否能有效区分正确分类与错误分类的样本,判断不确定性建模算法的好坏。除了包括与OOD 检测类似的常用指标AUROC 和 AUPRC。还有Expected Calibration Error (ECE)\cite{guo2017calibration},衡量模型预测置信度与实际准确率的偏差。Expected Calibration Error (ECE) 是衡量分类模型输出概率与实际标签一致性的一个指标。一个校准良好的模型,其预测概率应当与实际结果的发生频率一致。具体地,若模型预测概率为 \( p \),那么在长期实验中,模型预测为 \( p \) 的样本中,有 \( p \) 的比例应该属于正类。下面是ECE的计算方法:


1. 分组:
   将预测概率 \( p \) 分成若干个区间,例如: [0, 0.1), [0.1, 0.2), \dots, [0.9, 1.0] 这些区间可以根据需要调整。

2. 计算每个区间的校准误差:
   对于第 \( k \) 个区间 \( [p_k, p_{k+1}) \),计算该区间内所有样本的平均预测概率 \( \hat{p}_k \) 和真实标签的正确率 \( \text{accuracy}_k \)
   % \[
   % \hat{p}_k = \frac{1}{|S_k|} \sum_{i \in S_k} p_i
   % \]
   % \[
   % \text{accuracy}_k = \frac{1}{|S_k|} \sum_{i \in S_k} y_i
   % \]
   % 其中,\( S_k \) 是属于第 \( k \) 个区间的样本集合,\( p_i \) 是样本 \( i \) 的预测概率,\( y_i \) 是样本 \( i \) 的真实标签。

3. 计算校准误差:对每个区间计算校准误差,并计算所有区间的加权平均值:
\[
   \text{ECE} = \sum_k \frac{|S_k|}{N} | \hat{p}_k - \text{accuracy}_k |
   \]
   其中,\( N \) 是样本总数,\( |S_k| \) 是第 \( k \) 区间内样本的数量。


当 \(\text{ECE} = 0\) 时,表示模型是完全校准的,预测概率与实际频率完全一致。当 \(\text{ECE}\) 较大时,表示模型存在较大的校准误差,其概率输出与实际标签的匹配不佳。





\subsection{对抗样本检测}

对抗样本(Adversarial Examples) 是指通过对输入数据添加精细、刻意设计的微小扰动,使得机器学习模型对这些输入产生误判的样本。这些扰动通常对人类来说是不可察觉的,但会显著影响模型的预测结果。对抗样本检测的目标是识别经过精心设计以欺骗模型的样本。理想情况下,模型对对抗样本的预测应表现出更高的不确定性,所以使用不确定性作为对抗样本检测的指标,对抗样本识别的效果越好,不确定性建模的效果越好。对抗样本检测作为代理任务,评估不确定性建模方法好坏的思路是:首先使用一定的攻击算法生成一批对抗样本(如 FGSM\cite{goodfellow2015explaining} 、BIM\cite{kurakin2016adversarial}、PGD\cite{madry2017towards} 攻击),然后与正常样本一起输入模型并计算不确定性,检测不确定性是否能区分对抗样本与正常样本。衡量指标与 OOD 检测相同,主要使用 AUROC 和 AUPRC 来评估,AUROC和AUPRC两个指标越高,说明不确定性建模的效果越好。


\subsection{主动学习}


主动学习(Active Learning)\cite{ren2021survey}\cite{settles2009active}是一种机器学习方法,其中模型通过选择最具信息量的样本进行标注,以便在最少的标注成本下获得最大的学习效果。与传统的监督学习不同,主动学习不依赖于全部标注数据,而是通过一个学习算法主动选择样本,从而最大化模型的性能。一般的主动学习流程见\ref{fig:al}
\begin{figure}[h]
    \captionsetup{font=small, justification=centering}
    \centering
    \includegraphics[width=0.9\linewidth]{assets/activelearning.png}
    \caption{主动学习一般的流程\cite{ren2021survey}}
    \label{fig:al}
\end{figure}

在有限标注预算下,主动学习通过选择具有高不确定性的样本来优化标注数据集的选择策略。一个好的不确定性建模方法应能快速有效挑选提升模型性能的样本。通过迭代方式模拟主动学习过程,每轮根据不确定性选择样本进行标注,在固定训练样本规模下,通过主动学习策略挑选的样本训练模型,观察训练后的性能提升,如果依照不确定性挑选的样本能以较少标注样本达到目标性能,这种不确定性建模方法是良好的。主动学习评估不确定性建模方法的基本工作流程如下:

\begin{enumerate}[nosep]
    \item \textbf{初始化数据集}:首先从未标注的数据池中随机选择一小部分样本,并对这些样本进行标注,构成初始训练集。
    \item \textbf{模型训练}:使用初始训练集训练模型。
    \item \textbf{选择样本}:根据模型的当前性能,从未标注数据池中选择最能提高模型性能的样本。选择策略是依据不确定性采样,从候选池中选择模型最不确定的样本进行标注。
    \item \textbf{标注和迭代}:对选择的样本进行标注,加入到训练集中,然后重新训练模型。重复选择样本和训练的过程,直到满足某个停止条件(例如,达到一定的精度或标注次数)。
\end{enumerate}

\section{本章小结}
本章主要围绕不确定性的研究工作与理论展开讨论,分别从不确定性的建模方法、计算方法以及评估方法三个角度进行阐述。首先,介绍了几类主流的不确定性建模方法,包括贝叶斯神经网络、集成方法、测试时数据增强方法以及单一确定性神经网络建模,并详细对比了它们在优缺点。其次,本章介绍了分类与回归任务中不确定性计算的具体方式,例如分类任务中基于Softmax输出的预测熵、互信息和回归任务中的方差及预测区间长度等。最后指出由于标签信息的缺乏,模型的不确定性评估多依赖代理任务的表现,而非直接评估方法。整章通过分析总结不确定性建模领域的重要研究进展与挑战,为后续工作奠定了理论基础。


\chapter{基于输入扰动的改进}


\section{引言}
基于输入扰动的改进

\section{梯度空间的分析}
梯度空间的分析,统计关于输入/特征图的梯度响应图

\begin{figure}[H]
    \centering
    \includegraphics[width=0.75\linewidth]{assets/3-1.png}
    \caption{ResNet50 ,训练集 CIFAR10 vs OOD数据集SVHN
}
    \label{fig:enter-label}
\end{figure}

\begin{figure}[H]
    \centering
    \includegraphics[width=0.75\linewidth]{assets/3-2.png}
    \caption{VIT,训练集 CIFAR10 vs OOD数据集SVHN
}
    \label{fig:enter-label}
\end{figure}

\subsection{GradNorm的分析}
直接使用gardNorm作为不确定性的度量
\begin{table}[H]
	\centering
	\resizebox{\linewidth}{!}{
		\begin{tabular}{ |P{3cm}|P{3cm}|P{3cm}|  }
			\hline
			OOD dataset& auroc($\uparrow$) & auprc ($\uparrow$) \\
			\hline
			svhn & 0.9480 & \textbf{0.9613} \\
			lsun &   \textbf{0.9287} & \textbf{0.9444} \\
			cifar100 & \textbf{0.9058} & \textbf{0.9242} \\
			mnist   & \textbf{0.9665 }&\textbf{ 0.9770} \\
			\hline
			
			svhn+ip&\textbf{ 0.9493} & 0.9599\\
			lsun+ip &  0.8990 & 0.9137 \\
			cifar100+ip& 0.8871 & 0.9027\\
			mnist+ip  &  0.9481 & 0.9550\\
			\hline
		\end{tabular}
	}
	\caption{
		resnet50+cafar10,accuracy=0.9489,gradNorm
	}
\end{table}


\begin{table}[H]
	\centering
	\resizebox{\linewidth}{!}{	
		\begin{tabular}{ |P{3cm}|P{3cm}|P{3cm}|  }
			\hline
			OOD dataset& auroc($\uparrow$) & auprc ($\uparrow$) \\
			\hline
			svhn & 0.9293 & 0.9516 \\
			lsun &   \textbf{0.9436} & \textbf{0.9563} \\
			cifar100 & \textbf{0.9426} & \textbf{0.9480} \\
			mnist   & \textbf{0.9550 }&\textbf{ 0.9680} \\
			\hline
			
			svhn+ip&\textbf{ 0.9467} &\textbf{ 0.9633}\\
			lsun+ip &  0.9175 & 0.9469 \\
			cifar100+ip& 0.9086 & 0.9317\\
			mnist+ip  &  0.9113 & 0.9286\\
			\hline
		\end{tabular}
	}
	\caption{
		vit+cafar10,accuracy=0.9600,gradNorm
	}
\end{table}


\section{输入扰动}
加入输入扰动
加入输入扰动,对比Uncertainty的分布
\begin{figure}[H]
    \centering
    \includegraphics[width=0.75\linewidth]{assets/3-3.png}
    \caption{ResNet50 ,训练集 CIFAR10 vs OOD数据集SVHN}
    \label{fig:enter-label}
\end{figure}
\begin{figure}[H]
    \centering
    \includegraphics[width=0.75\linewidth]{assets/3-4.png}
    \caption{VIT ,训练集 CIFAR10 vs OOD数据集SVHN}
    \label{fig:enter-label}
\end{figure}

\subsection{算法流程图}
\begin{algorithm}[H]
	\caption{基于输入扰动的概率密度建模的模型不确定性算法}
	\label{alg:1}
	
	\begin{algorithmic}[1]
		\Require 训练集: $(X,Y)$ 
		\Require 谱归一化的高维特征提取网络: $f_{\theta}:x \rightarrow \mathbf{R}^d $ 
		\Require GMM模型: $q(z) = \sum_{y}q(z|y=c)q(y=c)$
		
		\\
		\Procedure {1.训练阶段}{}
		\State 在训练数据集数据集上训练网络$f_{\theta}$
		\For {属于类别c的样本}
		\State $\mu_{c}=\frac{1}{|x_c|}f_{\theta}(x_c)$
		\State $\Sigma_c = \frac{1}{|x_c|-1}(f_{\theta}(x_c)-\mu_c)(f_{\theta}(x_c)-\mu_c)^T$
		\State q(y=c)=$\frac{|X_C|}{|X|}$
		\EndFor
		\EndProcedure

		
		\\
		\Procedure {2.模型不确定性计算}{}
		\State $z=f_{\theta}(x)$
		\State  $q(z) = \sum_{y}q(z|y=c)q(y=c)$,其中$q(z)\sim N(\mu_c,\Sigma_c)$
		\State $\tilde{x}=x+\epsilon* \cdot sign(\nabla_x \log q(z)))$,其中$\epsilon$通过grid search调参
		\State $\tilde{z}=f_{\theta}(\tilde{x})$
		\State 计算模型不确定性 $Uncertainty(x) = \sum_{y}q(\tilde{z}|y=c)q(y=c)$,其中$q(\tilde{z})\sim N(\mu_c,\Sigma_c)$
		\EndProcedure
	\end{algorithmic}
\end{algorithm}



\section{实验结果与分析}
在OOD检测任务上,对抗样本检测任务上,主动学习任务上评估。
\subsection{OOD检测任务上评估}



\subsubsection{Cifar加入输入扰动前后对比}
\begin{table}[H]
	\centering
	\resizebox{\linewidth}{!}{
		
		\begin{tabular}{ |P{3cm}|P{3cm}|P{3cm}|  }
			\hline
			OOD dataset& auroc($\uparrow$)  & auprc($\uparrow$)  \\
			\hline
			svhn & \textbf{0.9221} & \textbf{0.9432} \\
			lsun &   0.9363 & 0.9528 \\
			cifar100 & 0.8861 & \textbf{0.9028} \\
			mnist   & 0.9189 & 0.9369 \\
			tiny-imagenet & 0.9318 & 0.9479 \\
			\hline
			
			svhn+ip& 0.9216 & 0.9411\\
			lsun+ip & \textbf{ 0.9394} & \textbf{0.9533} \\
			cifar100+ip& \textbf{ 0.8885} & 0.9009\\
			mnist+ip  &  \textbf{0.9637} & \textbf{0.9666}\\
			tiny-imagenet+ip& \textbf{ 0.9397} & \textbf{0.9493}\\
			\hline
		\end{tabular}
	}
	\caption{
		vgg16+cafar10,accuracy=0.9405
	}
\end{table}



\begin{table}[H]
	\centering
	\resizebox{\linewidth}{!}{
		
		\begin{tabular}{ |P{3cm}|P{3cm}|P{3cm}|  }
			\hline
			OOD dataset& auroc($\uparrow$) & auprc ($\uparrow$) \\
			\hline
			svhn & 0.9480 & 0.9613 \\
			lsun &   0.9365 & 0.9497 \\
			cifar100 & 0.9068 & 0.9257 \\
			mnist   & 0.9774 & 0.9845 \\
			tiny-imagenet & 0.9469 & 0.9580 \\
			\hline
			
			svhn+ip&\textbf{ 0.9735} & \textbf{0.9769}\\
			lsun+ip & \textbf{ 0.9671} & \textbf{0.9716} \\
			cifar100+ip& \textbf{ 0.9171} & \textbf{0.9372}\\
			mnist+ip  &  \textbf{0.9939} & \textbf{0.9957}\\
			tiny-imagenet+ip& \textbf{ 0.9676} & \textbf{0.9731}\\
			\hline
		\end{tabular}
	}
	\caption{
		resnet50+cafar10,D=2048,accuracy=0.9489
	}
\end{table}

\begin{table}[H]
	\centering
	\resizebox{\linewidth}{!}{
		
		\begin{tabular}{ |P{3cm}|P{3cm}|P{3cm}|  }
			\hline
			OOD dataset& auroc($\uparrow$)  & auprc($\uparrow$)  \\
			\hline
			svhn & 0.9685 & 0.9768 \\
			lsun &   0.9720 & 0.9788 \\
			cifar100 & 0.9400 & 0.9514 \\
			mnist   & 0.9906 & 0.9929 \\
			tiny-imagenet & 0.9747 & 0.9803 \\
			\hline
			
			svhn+ip& \textbf{0.9891} & \textbf{0.9904}\\
			lsun+ip & \textbf{ 0.9811} & \textbf{0.9830} \\
			cifar100+ip& \textbf{ 0.9497} & \textbf{0.9585}\\
			mnist+ip  &  \textbf{0.9978} & \textbf{0.9982}\\
			tiny-imagenet+ip& \textbf{ 0.9836} & \textbf{0.9861}\\
			\hline
		\end{tabular}
	}
	\caption{
		wideResnet+cafar10,accuracy=0.9650
	}
\end{table}


\begin{table}[H]
	\centering
	\resizebox{\linewidth}{!}{
		
		\begin{tabular}{ |P{3cm}|P{3cm}|P{3cm}|  }
			\hline
			OOD dataset& auroc($\uparrow$)  & auprc($\uparrow$)  \\
			\hline
			svhn & 0.9293 & 0.9516 \\
			lsun &   0.9436 & 0.9563 \\
			cifar100 & 0.9426 & 0.9480 \\
			mnist   & 0.9550 & 0.9680 \\
			tiny-imagenet & 0.9330 & 0.9367 \\
			\hline
			
			svhn+ip& \textbf{0.9728} & \textbf{0.9793}\\
			lsun+ip & \textbf{ 0.9688} & \textbf{0.9740} \\
			cifar100+ip& \textbf{ 0.9463} & \textbf{0.9495}\\
			mnist+ip  &  \textbf{0.9904} & \textbf{0.9926}\\
			tiny-imagenet+ip& \textbf{ 0.9546} & \textbf{0.9545}\\
			\hline
		\end{tabular}
	}
	\caption{
		vit+cafar10,accuracy=0.9600
	}
\end{table}


\subsubsection{mnist加入输入扰动前后对比}
\begin{table}[H]
	\centering
	\resizebox{\linewidth}{!}{	
		\begin{tabular}{ |P{3cm}|P{3cm}|P{3cm}|  }
			\hline
			OOD dataset& auroc($\uparrow$)  & auprc($\uparrow$)  \\
			\hline
			cifar10 & 0.9949 & 0.9964 \\
			fashionmnist &   0.9914 & 0.9933 \\
			fer2013 & 0.9952 & 0.9969 \\
			lsun   & 0.9946 & 0.9962 \\
			cifar100 & 0.9945 & 0.9961 \\
			svhn & 0.9941 & 0.9960 \\
			\hline
			
			cifar10+ip&\textbf{ 0.9965} &\textbf{ 0.9974}\\
			fashionmnist+ip & \textbf{ 0.9921} & \textbf{0.9936} \\
			fer2013+ip& \textbf{ 0.9970} & \textbf{0.9978}\\
			svhn+ip  &  \textbf{0.9963} & \textbf{0.9972}\\
			cifar10+ip& \textbf{ 0.9961} & \textbf{0.9970}\\
			svhn+ip & 0.9964 & 0.9973 \\
			\hline
		\end{tabular}
	}
	\caption{
		vgg16+mnist,accuracy=0.9882
	}
\end{table}


\begin{table}[H]
	\centering
	\resizebox{\linewidth}{!}{
		\begin{tabular}{ |P{3cm}|P{3cm}|P{3cm}|  }
			\hline
			OOD dataset& auroc($\uparrow$)  & auprc($\uparrow$)  \\
			\hline
			cifar10 & 0.9949 & 0.9964 \\
			fashionmnist &   0.9914 & 0.9933 \\
			fer2013 & 0.9952 & 0.9969 \\
			lsun   & 0.9946 & 0.9962 \\
			cifar100 & 0.9945 & 0.9961 \\
			svhn & 0.9941 & 0.9960 \\
			\hline
			
			cifar10+ip&\textbf{ 0.9965} &\textbf{ 0.9974}\\
			fashionmnist+ip & \textbf{ 0.9921} & \textbf{0.9936} \\
			fer2013+ip& \textbf{ 0.9970} & \textbf{0.9978}\\
			svhn+ip  &  \textbf{0.9963} & \textbf{0.9972}\\
			cifar10+ip& \textbf{ 0.9961} & \textbf{0.9970}\\
			svhn+ip & 0.9964 & 0.9973 \\
			\hline
		\end{tabular}
	}
	\caption{
		resnet50+mnist,accuracy=0.9870
	}
\end{table}

\subsubsection{各种方法的对比}
\begin{table}[H]
		\captionsetup{labelformat=empty}
	\centering
	\resizebox{\linewidth}{!}{
\begin{tabular}{|c|c|c|c|}
\hline
Method & OOD Dataset &  AUROC($\uparrow$) & AUROC($\uparrow$) \\
\hline
\multirow{3}{*}{baseline} 
& cifar100 & 0.8786 & 0.8774 \\
& lsun & 0.8952 & 0.8997 \\
& mnist & 0.9434 & 0.9524 \\
& svhn & 0.9271 & 0.9405 \\
\hline
\multirow{3}{*}{ensemble} 
& cifar100 & 0.9244 & 0.9166 \\
& lsun & 0.9724 & 0.9685 \\
& mnist & 0.9735 & 0.9706 \\
& svhn & 0.9667 & 0.9599 \\
\hline
\multirow{3}{*}{ddu}
& cifar100 & 0.9068 & 0.9257 \\
& lsun & 0.9365 & 0.9497 \\
& mnist & 0.9774 & 0.9845 \\
& svhn & 0.9480 & 0.9613 \\
\hline
\multirow{3}{*}{ddu+ip} 
& cifar100 & 0.9171 & 0.9312 \\
& lsun & 0.9671 & 0.9716 \\
& mnist & 0.9939 & 0.9957 \\
& svhn & 0.9735 & 0.9769 \\
\hline
\end{tabular}
}
	\caption{
	resnet50+cafar10,D=2048,accuracy=0.9540
}
\end{table}

\subsubsection{GMM vs KDE 实验结果}
\begin{table}[H]
	\centering
	\resizebox{\linewidth}{!}{
		\begin{tabular}{|P{3cm}|P{3cm}|P{3cm}|}
			\hline
			OOD dataset& auroc($\uparrow$)  & auprc($\uparrow$)  \\
			\hline
			svhn+gmm & \textbf{0.9458} & \textbf{0.9564} \\
			lsun+gmm &   \textbf{0.9287} & \textbf{0.9444} \\
			cifar100+gmm & \textbf{ 0.9058 }& \textbf{0.9242} \\
			mnist +gmm  & \textbf{0.9665} & \textbf{0.9770} \\
			tiny-imagenet+gmm & \textbf{0.8932} & \textbf{0.9112} \\
			\hline
			
			svhn+kde& 0.7924 & 0.8365\\
			lsun+kde &  0.7894 & 0.8264 \\
			cifar100+kde& 0.7772 & 0.8097\\
			mnist+kde  &  0.8364 & 0.8684 \\
			tiny-imagenet+kde& 0.7653 & 0.7950\\
			\hline
		\end{tabular}
	}
	\caption{
		resnet50+cafar10,GMM vs KDE ,Dimension=2048,accuracy=0.9489
	}
\end{table}


\begin{table}[H]
	\centering
	\resizebox{\linewidth}{!}{
		
		\begin{tabular}{ |P{3cm}|P{3cm}|P{3cm}|  }
			\hline
			OOD dataset& auroc($\uparrow$)  & auprc($\uparrow$)  \\
			\hline
			svhn+gmm & 0.9320 & 0.9530 \\
			lsun+gmm &   \textbf{0.9302} & \textbf{0.9471} \\
			cifar100+gmm & \textbf{ 0.9014 }& \textbf{0.9156} \\
			mnist +gmm  & 0.9358 & 0.9563 \\
			tiny-imagenet+gmm & \textbf{0.9026} & \textbf{0.9162} \\
			\hline
			
			svhn+kde& \textbf{0.9528} & \textbf{0.9620}\\
			lsun+kde &  0.9179 & 0.9299 \\
			cifar100+kde& 0.8923 & 0.9022\\
			mnist+kde  &  \textbf{0.9654} & \textbf{0.9718} \\
			tiny-imagenet+kde& 0.8975 & 0.8975\\
			\hline
		\end{tabular}
	}
	\caption{
		resnet50+cafar10,GMM vs KDE ,Dimension=1024,accuracy=0.9504
	}
\end{table}

\begin{table}[H]
	\centering
	\resizebox{\linewidth}{!}{
		
		\begin{tabular}{ |P{3cm}|P{3cm}|P{3cm}|  }
			\hline
			OOD dataset& auroc($\uparrow$)  & auprc($\uparrow$)  \\
			\hline
			svhn+gmm & 0.9364 & 0.9553 \\
			lsun+gmm &   \textbf{0.9322} & \textbf{0.9474} \\
			cifar100+gmm & \textbf{ 0.9028 }& \textbf{0.9156} \\
			mnist +gmm  & 0.9378 & 0.9567 \\
			tiny-imagenet+gmm & \textbf{0.9038} & \textbf{0.9133} \\
			\hline
			
			svhn+kde& \textbf{0.9484} & \textbf{0.9584}\\
			lsun+kde &  0.9211 & 0.9312 \\
			cifar100+kde& 0.8957 & 0.9019\\
			mnist+kde  &  \textbf{0.9662} & \textbf{0.9732} \\
			tiny-imagenet+kde& 0.9018 & 0.8992\\
			\hline
		\end{tabular}
	}
	\caption{
		resnet50+cafar10,GMM vs KDE ,Dimension=512,accuracy=0.9520
	}
\end{table}

\begin{table}[H]
	\centering
	\resizebox{\linewidth}{!}{
		
		\begin{tabular}{ |P{3cm}|P{3cm}|P{3cm}|  }
			\hline
			OOD dataset& auroc($\uparrow$)  & auprc($\uparrow$)  \\
			\hline
			svhn+gmm & 0.9418 & 0.9594 \\
			lsun+gmm &   \textbf{0.9277} & \textbf{0.9417} \\
			cifar100+gmm &  0.8983 & 0.9073 \\
			mnist +gmm  & 0.9450 & 0.9628 \\
			tiny-imagenet+gmm & 0.9013 & 0.9066 \\
			\hline
			
			svhn+kde& \textbf{0.9569} & \textbf{0.9659}\\
			lsun+kde &  0.9230 & 0.9399 \\
			cifar100+kde& \textbf{0.8992} & \textbf{0.9090}\\
			mnist+kde  &  \textbf{0.9730} & \textbf{0.9790} \\
			tiny-imagenet+kde& \textbf{0.9097} & \textbf{0.9099}\\
			\hline
		\end{tabular}
	}
	\caption{
		resnet50+cafar10,GMM vs KDE ,Dimension=256,accuracy=0.9540
	}
\end{table}


\begin{table}[H]
	\centering
	\resizebox{\linewidth}{!}{
		
		\begin{tabular}{ |P{3cm}|P{3cm}|P{3cm}|  }
			\hline
			OOD dataset& auroc($\uparrow$)  & auprc($\uparrow$)  \\
			\hline
			svhn+gmm & \textbf{0.9222} & \textbf{0.9433} \\
			lsun+gmm &   \textbf{0.9077} & \textbf{0.9204} \\
			cifar100+gmm & \textbf{ 0.8848 }& \textbf{0.9018} \\
			mnist +gmm  & 0.9141 & 0.9308 \\
			tiny-imagenet+gmm & \textbf{0.8807} & \textbf{0.8847} \\
			\hline
			
			svhn+kde& \textbf{0.9230} & \textbf{0.9362}\\
			lsun+kde &  0.9047 & 0.9137 \\
			cifar100+kde& 0.8778 & 0.8845\\
			mnist+kde  &  \textbf{0.9377}& \textbf{0.9456} \\
			tiny-imagenet+kde& 0.8750 & 0.8643\\
			\hline
		\end{tabular}
	}
	\caption{
		vgg16+cafar10,GMM vs KDE ,Dimension=512,accuracy=0.9489
	}
\end{table}




\subsection{对抗样本检测任务上评估}

\begin{table}[H]
	\captionsetup{labelformat=empty}
	\centering
	\begin{tabular}{|ll|rr|}
		\hline
		\multicolumn{2}{|c|}{Input} & \multicolumn{2}{c|}{Result}  \\
		\hline
		Method      & Attack Method  & AUROC($\uparrow$)  & AUROC($\uparrow$)   \\
		\hline
		\multirow{4}{*}{DDU} 
		& FGSM & 0.8428 & 0.8394 \\
		& BIM  & 0.8070 & 0.8816 \\
		& PGD  & 0.7420 & 0.8021 \\
		\hline
		\multirow{4}{*}{GMM+Input Perturbation} 
		& FGSM & \textbf{0.8882} & \textbf{0.9737} \\
		& BIM   & \textbf{0.8559} & \textbf{0.9686} \\
		& PGD   & \textbf{0.8433} & \textbf{0.9700} \\
		\hline
	\end{tabular}
	\caption{VGG16 + CIFAR10, 对抗样本的检测任务}
\end{table}

\subsection{主动学习任务上评估}
\begin{figure}[H]
    \centering
    \includegraphics[width=0.75\linewidth]{assets/3-11.png}
    \caption{主动学习任务上对比}
    \label{fig:enter-label}
\end{figure}

\section{本章小结}
结论



\chapter{基于辅助损失函数联合训练的改进}
\section{引言}
为了特征空间的类内紧密性和类间可分性,可以引入辅助损失函数联合训练。
\section{特征空间的分析}
tsne可视化
\begin{figure}[H]
    \centering
    \includegraphics[width=0.8\linewidth]{assets/4-1.png}
    \caption{MNIST特征空间的tsne可视化}
    \label{fig:enter-label}
\end{figure}
\section{辅助损失函数联合训练}
centerloss和contrastive center loss

\begin{figure}[H]
    \centering
    \includegraphics[width=0.8\linewidth]{assets/4-2.png}
    \caption{CIFAR10特征空间的tsne可视化}
    \label{fig:enter-label}
\end{figure}

\section{实验结果与分析}
\subsection{OOD检测任务上评估}
\begin{table}[H]
	\captionsetup{labelformat=empty}
	\centering
	\renewcommand{\arraystretch}{1.2} % 增加行高,提高可读性
	\setlength{\tabcolsep}{8pt} % 调整列间距
	\begin{tabular}{|c|c|c|c|}
		\hline
		\multicolumn{2}{|c|}{\textbf{实验设置}} & \multicolumn{2}{c|}{\textbf{实验结果}} \\ 
		\hline
		\textbf{训练策略} & \textbf{OOD Dataset} & \textbf{ AUROC($\uparrow$)} & \textbf{ AUPRC ($\uparrow$)} \\ 
		\hline
		\multirow{4}{*}{CE loss} 
		& svhn      & 0.9195 & 0.9412 \\ 
		& lsun      & 0.9144 & \textbf{ 0.9317} \\ 
		& cifar100  & 0.8865 & 0.8966 \\ 
		& mnist     & 0.9240 & 0.9426 \\ 
		\hline
		\multirow{4}{*}{CE loss +ContrastiveCenterLoss}
		& svhn      & \textbf{ 0.9332} & \textbf{0.9517} \\ 
		& lsun      & \textbf{ 0.9221} & 0.9270 \\ 
		& cifar100  & \textbf{ 0.8952} & \textbf{0.9065} \\ 
		& mnist     & \textbf{0.9395} & \textbf{0.9579} \\ 
		\hline
	\end{tabular}
	\caption{实验设置:Vgg16+cafar10,D=512,accuracy=0.9438}
\end{table}



\begin{table}[H]
	\captionsetup{labelformat=empty}
	\centering
	\renewcommand{\arraystretch}{1.2} % 增加行高,提高可读性
	\setlength{\tabcolsep}{8pt} % 调整列间距
	\begin{tabular}{|c|c|c|c|}
		\hline
		\multicolumn{2}{|c|}{\textbf{实验设置}} & \multicolumn{2}{c|}{\textbf{实验结果}} \\ 
		\hline
		\textbf{训练策略} & \textbf{OOD Dataset} & \textbf{ AUROC($\uparrow$)} & \textbf{ AUPRC ($\uparrow$)} \\ 
		\hline
		\multirow{4}{*}{CE loss} 
		&	svhn & 0.9260 & 0.9437 \\
		&	lsun &   0.9290 &\textbf{0.9445} \\
		&	cifar100 & 0.9036 & 0.9124 \\
		&	mnist   & 0.9176 & 0.9268 \\
		\hline
		\multirow{4}{*}{CE loss + ContrastiveCenterLoss}
		&	svhn&\textbf{0.9410} & \textbf{ 0.9554}\\
		&	lsun & \textbf{0.9329} & 0.9415 \\
		&	cifar100&  \textbf{0.9115} &\textbf{ 0.9177}\\
		&	mnist  &  \textbf{0.9501} & \textbf{0.9636}\\
		\hline
	\end{tabular}
	\caption{实验设置:ResNet18+cafar10,D=512,accuracy=0.9438}
\end{table}


\begin{table}[H]
	\captionsetup{labelformat=empty}
	\centering
	\renewcommand{\arraystretch}{1.2} % 增加行高,提高可读性
	\setlength{\tabcolsep}{8pt} % 调整列间距
	\begin{tabular}{|c|c|c|c|}
		\hline
		\multicolumn{2}{|c|}{\textbf{实验设置}} & \multicolumn{2}{c|}{\textbf{实验结果}} \\ 
		\hline
		\textbf{训练策略} & \textbf{OOD Dataset} & \textbf{ AUROC($\uparrow$)} & \textbf{ AUPRC ($\uparrow$)} \\ 
		\hline
		\multirow{4}{*}{CE loss} 
		&	svhn & 0.9007 & 0.9008 \\
		&	lsun &   0.9727 & 0.9716 \\
		&	cifar100 & 0.8960 & 0.8966 \\
		&	mnist   & 0.8913 & 0.9123 \\
		\hline
		\multirow{4}{*}{CE loss + ContrastiveCenterLoss}
		&	svhn&\textbf{0.9874} & \textbf{ 0.9910}\\
		&	lsun & \textbf{0.9891} & \textbf{0.9916} \\
		&	cifar100&  \textbf{0.9668} &\textbf{ 0.9704}\\
		&	mnist  &  \textbf{0.9930} & \textbf{0.9939}\\
		\hline
	\end{tabular}
	\caption{实验设置:VIT+cafar10,accuracy=0.9780}
\end{table}

\subsection{对抗样本检测任务上评估}
\section{本章小结}
\chapter{总结与展望}
\section{总结与反思}
% 本文研究内容主要是基于高维特征概率密度建模神经网络的模型不确定性。前两章主要围绕不确定性的研究工作与理论展开讨论,分别从不确定性的建模方法、计算方法以及评估方法三个角度进行阐述。首先,介绍了几类主流的不确定性建模方法,包括贝叶斯神经网络、集成方法、测试时数据增强方法以及单一确定性神经网络方法,并详细对比了它们在优缺点、适用场景及计算复杂度上的异同。其次,探讨了分类与回归任务中不确定性计算的具体方式,例如基于Softmax输出的预测熵、MC Dropout的熵分解及贝叶斯推断方法的应用等。最后,指出由于标签信息的缺乏,模型的不确定性评估多依赖代理任务的表现,而非直接评估方法。

% 第三章在高维特征概率密度建模的基础上,提出了一种基于输入扰动的改进方法,用于提升模型不确定性量化的能力。研究发现,样本的不确定性与梯度空间的响应具有显著相关性,OOD样本在梯度空间中的响应普遍较高,而训练集样本的响应相对较低。这一发现启发通过输入扰动来增强特征分布的捕捉能力。通过将梯度信息与概率密度建模相结合,该方法不仅为神经网络不确定性量化提供了新的思路,也为实际应用中模型的可靠性提升奠定了基础。然而,输入扰动方法的效果在一定程度上依赖于超参数的选择以及具体的数据分布特性。因此,未来工作可以进一步探讨自适应的扰动生成机制,以实现更为通用且稳健的不确定性建模方法。此外,可以尝试将本章方法与其他先进技术相结合,进一步挖掘梯度空间与特征分布的潜在关联,为神经网络的解释性与鲁棒性研究提供更有力的支持。不同类型的扰动如何影响特征分布,以及扰动对模型学习过程的长期影响,这些问题仍待深入探讨。

% 第四章主要研究了基于特征空间优化的神经网络不确定性建模方法,旨在提升分类性能和预测可靠性。通过引入度量学习技术和提出创新辅助损失函数AuxLoss,优化了特征空间的类内紧密性和类间可分性,有效增强了模型在不同任务中的适应能力与鲁棒性。此外,通过结合PCA降维和高斯判别分析模型,解决了高维数据的存储复杂性问题,实现了高效的不确定性量化。这些方法在MNIST和CIFAR-10等数据集上的实验验证表明,AuxLoss显著提升了模型的分类性能和特征分布质量,而PCA和GDA结合策略进一步提高了计算效率和存储性能。尽管如此,本研究在高维多模态数据、实时计算优化和理论机制深化等方面仍存在进一步探索的空间。未来工作可致力于设计更加自适应的损失函数,以应对复杂任务中的动态变化;在计算效率方面,可结合随机投影、稀疏建模或硬件加速技术,优化高维数据处理流程;此外,针对多模态场景,可以研究特征融合优化策略,从而提升模型在跨领域任务中的通用性与可靠性。同时,通过与输入扰动、对抗训练等方法的结合,进一步完善模型的不确定性建模体系。在理论方面,可基于信息论或概率分布理论,深入揭示特征空间优化与模型性能的内在联系。综上所述,本章研究为神经网络的不确定性建模和特征空间优化提供了有力支持,为后续研究和实际应用奠定了坚实基础。

本文围绕神经网络模型的不确定性建模展开研究,首先综述了当前主流的不确定性建模方法,包括贝叶斯神经网络、模型集成、测试时数据增强以及单一确定性神经网络方法,并分析了各自的优缺点。贝叶斯神经网络、模型集成及测试时数据增强方法在计算复杂度较高,建模效率相对较低,为了更高效地建模神经网络的不确定性,本文将研究重心放在了单一确定性-建模方法上,重点探讨了如何在高维特征概率密度估计的框架下提升不确定性建模的效果。针对现有方法在模型不确定性建模效果上的局限性,本文提出了两种改进方法:

其一是基于输入扰动的改进策略:本文发现分布内样本与分布外样本在梯度空间中存在显著差异,分布外样本的梯度响应更大,并通过理论分析对此现象进行了理论上的分析。基于这一结论,提出了引入梯度相关扰动的方法,以增强两类样本在概率密度分布上的差异性。实验结果表明,该策略在OOD检测、对抗样本识别等任务中显著提升了模型不确定性的建模效果。

其二是基于联合训练的改进策略:本文分析了特征空间结构对不确定性建模效果的影响,指出提升类内紧密性和类间可分性有助于提高基于高维特征概率密度估计的模型不确定性建模能力。为此,本文设计了一种全新的辅助损失函数,并结合交叉熵损失进行联合训练,以优化特征空间结构。实验结果表明,该方法能够在不同任务上显著提升模型不确定性建模的性能。此外,为降低高维特征带来的存储开销,本文提出利用主成分分析(PCA)进行降维,不仅减少存储需求,同时提高了建模效率和效果。

本文的主要贡献总结如下:
\begin{itemize}
    \item 发现并解释了分布外样本在梯度空间上的响应更大,分布内样本在梯度空间上的响应更小。
    \item 提出基于梯度的输入扰动策略,以提升基于概率密度估计的模型不确定性建模的效果。
    \item 设计新的辅助损失函数,提高特征空间的类内紧密性和类间可分性,从而提升基于概率密度估计的模型不确定性的建模效果。
    \item 采用PCA降维,降低存储复杂度,同时提升建模效率。
\end{itemize}

尽管本文在神经网络不确定性建模方面取得了一定进展,但仍存在以下值得进一步探讨的问题:

\begin{itemize}
    \item 计算复杂度与实时性问题: 虽然本文提出的方法在一定程度上提升了不确定性建模的精度,但计算梯度信息及特征优化仍然带来了额外的计算开销。在计算资源受限的环境(如嵌入式系统)中,如何进一步降低计算成本,同时保持良好的不确定性量化能力,是未来研究的重要方向。
    
    \item 方法的泛化性: 本文的研究主要围绕图像分类任务展开,然而,基于高维特征概率密度估计的建模不确定性的方法是否适用于其他任务(如目标检测、语义分割、自然语言处理等)仍需进一步验证。此外,不同神经网络架构对本文方法的适应性仍需深入探索,未来可进一步研究如何将所提出的策略推广到更广泛的深度学习应用场景。

    \item 全面的不确定性建模框架: 目前的不确定性量化方法主要依赖概率密度估计,而不确定性本质上是一个复杂概念,可能涉及数据噪声、模型结构及训练数据分布等多个因素。未来研究可考虑结合其他不确定性建模方法(如贝叶斯推断、信息论度量等),构建更加全面的不确定性量化框架。

    \item 理论分析的深入性: 本文通过实验和部分理论推导验证了所提出方法的有效性,但仍需更深入的数学分析,以更严格地刻画输入扰动、特征优化与不确定性量化之间的关系。此外,研究不同数据集、不同模型架构下的方法泛化性,并建立相应的理论框架,是未来研究的重要方向。
\end{itemize}

综上所述,本文针对单一确定性神经网络的不确定性建模问题,提出了两种基于高维特征概率密度估计的改进策略,并在多个任务上验证了其有效性。然而,在计算效率、泛化性及理论分析方面仍有进一步优化的空间。

\section{对神经网络不确定性研究的展望}

随着深度学习的广泛应用,神经网络的不确定性研究在模型安全性、鲁棒性和可靠性等领域展现出重要价值,未来对不确定性的研究可以从以下几个方面展开。

现有的不确定性建模方法,如贝叶斯神经网络和深度集成方法,往往面临计算成本高和难以扩展的问题。未来研究可以关注开发计算效率更高的模型,适用于大规模数据集和实时应用,构建能够同时捕获数据不确定性和模型不确定性的统一建模框架。

当前对不确定性模型的评估依赖于间接任务(如 OOD 检测),难以全面反映实际性能。未来可以探索真实任务驱动评估,设计更加贴近实际应用的不确定性评估任务。结合模型性能、不确定性分布的合理性和计算效率,建立综合评估体系。

随着任务复杂度增加,未来不确定性建模需要适配新的应用场景,例如:多模态数据,探索图像、文本和音频等多模态数据的不确定性建模方法。在线学习与自适应系统,开发能动态调整不确定性的模型,适应在线学习和环境变化。

近年来,大规模预训练模型GPT3\cite{brown2020language}、BERT\cite{devlin2019bert}、BLOOM\cite{scao2022bloom}等在多个领域取得显著进展。不确定性研究未来可以聚焦于预训练与微调阶段的不确定性,研究大规模模型在不同训练阶段中的不确定性特性,开发针对大规模模型的不确定性标定方法,分析大规模预训练模型在迁移学习和跨领域任务中的不确定性。

神经网络不确定性研究作为机器学习的重要方向,未来将进一步推动 AI 系统的可靠性与广泛应用。随着理论方法的深入发展和新兴应用场景的拓展,不确定性建模将成为构建智能、透明和可信 AI 系统的核心技术之一。






% 附录部分
\appendix
一些额外实验


\chapter{公式推导}
一些公式


% 后置部分包含参考文献、声明页(自动生成)等
\backmatter

% 打印参考文献列表
\printbibliography

\begin{acknowledgements}
  致谢内容
\end{acknowledgements}

\end{document}
