\documentclass[twocolumn]{ctexart}
\CTEXsetup[format={\Large\bfseries}]{section}%让section指令左对齐
\renewcommand{\abstractname}{}%去掉摘要头上的标题
\newcommand{\upcite}[1]{\textsuperscript{\textsuperscript{\cite{#1}}}}%右上角引用文献的命令
\usepackage[margin=2cm]{geometry}%调整页边距
\usepackage{pifont}%提供圆圈数字输入
\usepackage{graphicx}%插入图片
\usepackage{authblk}%作者、单位
\usepackage{amsmath, bm}%数学公式宏包
\usepackage{esint}%使重积分符号更加紧凑,必须加在amsmath后
\usepackage{amssymb}%特殊数学符号
\usepackage{caption}%图片标题处理
\usepackage{float}%处理图表浮动插入
\usepackage[section]{placeins}%防止图表浮动跨过section
\usepackage{subfigure}%插入多图时用子图显示的宏包
\pagestyle{plain}%页码
\setCJKfamilyfont{zhsong}[AutoFakeBold = {2.17}]{SimSun}
\setCJKmainfont{SimSun}[BoldFont=FandolSong-Bold]
\renewcommand*{\songti}{\CJKfamily{zhsong}}%定义新宋体命令
\setlength{\belowcaptionskip}{-2pt}
\begin{document}
	
	
	
	\zihao{5}%设置全文字号为五号
	\everymath{\displaystyle}%设置所有数学公式为displaystyle形式
	\abovedisplayshortskip=5pt%设置数学公式间距
	\belowdisplayshortskip=5pt
	\abovedisplayskip=5pt
	\belowdisplayskip=5pt
	\lineskiplimit=4pt
	\lineskip=4pt
	\title{\vspace{-2cm}{\heiti {\zihao{2}论文标题}} }%标题
	\date{}%不显示日期
	\author[1]{\zihao{-4}作者1 , 作者2 , 作者3\vspace{-1.5em}}%作者名称
	\affil[1]{\vspace{-6em}{{\zihao{6}{\kaishu (单位)}}}}%作者单位
	\twocolumn[
	\begin{@twocolumnfalse}
		\maketitle 
		\begin{abstract}
			\newgeometry{left=1.5cm, right=1.5cm}%调整摘要部分的页边距,与正文对齐
			\noindent{\zihao{-5}{\heiti 摘~~~要 }{\kaishu ~~~这是一段摘要。这是一段摘要。这是一段摘要。这是一段摘要。这是一段摘要。这是一段摘要。这是一段摘要。这是一段摘要。这是一段摘要。这是一段摘要。这是一段摘要。这是一段摘要。这是一段摘要。这是一段摘要。这是一段摘要。这是一段摘要。这是一段摘要。这是一段摘要。这是一段摘要。这是一段摘要。这是一段摘要。这是一段摘要。这是一段摘要。这是一段摘要。这是一段摘要。这是一段摘要。这是一段摘要。这是一段摘要。这是一段摘要。这是一段摘要。}}\\
			\noindent{\zihao{-5}\heiti 关键词 }~~~{\zihao{-5}\kaishu 关键词1~~~~关键词2~~~~关键词3}\\
			\\
		\end{abstract}
	\end{@twocolumnfalse}
	]%双栏环境下单栏的摘要
	
	\section{{\zihao{4}{\songti 第一节}}\vspace{-0.6em}}
	\subsection{{\songti 第一小节}\vspace{-0.6em}}
	
	这是一段正文。 这是一段正文。 这是一段正文。 这是一段正文。 这是一段正文。 这是一段正文。 这是一段正文。 这是一段正文。 这是一段正文。 这是一段正文。 这是一段正文。 这是一段正文。 这是一段正文。 这是一段正文。 这是一段正文。 这是一段正文。 这是一段正文。 这是一段正文。 这是一段正文。 这是一段正文。 这是一段正文。 这是一段正文。 这是一段正文。 这是一段正文。 这是一段正文。 这是一段正文。 这是一段正文。 这是一段正文。 这是一段正文。 这是一段正文。 这是一段正文。 这是一段正文。 这是一段正文。 这是一段正文。 这是一段正文。 这是一段正文。 这是一段正文。 这是一段正文。 这是一段正文。 这是一段正文。 这是一段正文。 这是一段正文。
	\subsection{{\songti 第二小节}\vspace{-0.6em}}
	这是一段数学公式
	\begin{equation}
		\begin{aligned}
			p=\frac{L^{2}}{G M m^{2}}, \quad \varepsilon=\sqrt{1+\frac{2 E L^{2}}{G^{2} M^{2} m^{3}}} \\
		\end{aligned}\tag{1.2.1}
	\end{equation}
	\begin{equation}
	\begin{aligned}
		\mathrm{~d} \theta=\frac{\mathrm{d} r / r^{2}}{\sqrt{\left(\dfrac{\varepsilon}{p}\right)^{2}-\left(\dfrac{1}{r}-\dfrac{1}{p}\right)^{2}}} .
	\end{aligned}\tag{1.2.2}
	\end{equation}
	\section{{\zihao{4}{\songti 第二节}}\vspace{-0.6em}}
	
	\subsection{{\songti 第一小节}\vspace{-0.6em}}
	
	\subsection{{\songti 第二小节}\vspace{-0.6em}}
	
	\section{{\zihao{4}{\songti 第三节}}\vspace{-0.6em}}
	
	\subsection{{\songti 第一小节}\vspace{-0.6em}}
	
	\subsection{{\songti 第二小节}\vspace{-0.6em}}
	
	\section{{\zihao{4}{\songti 第四节}}\vspace{-0.6em}}
	
	\subsection{{\songti 第一小节}\vspace{-0.6em}}
	
	\subsection{{\songti 第二小节}\vspace{-0.6em}}
	
	\section{{\zihao{4}{\songti 第五节}}\vspace{-0.6em}}
	
	\subsection{{\songti 第一小节}\vspace{-0.6em}}
	
	\subsection{{\songti 第二小节}\vspace{-0.6em}}
	
	%如果你会用BibTeX,请使用生成参考文献列表的命令
	%\bibliographystyle{unsrt}
	%\bibliography{ref} 
	%如果不会用就如下所示手打参考文献
	\noindent\textbf{\songti {\zihao{-3} 参考文献}}\\%参考文献
	$\left[1\right]$王跃洲. 基于力学近似模型的大展弦比机翼结构优化设计[D].哈尔滨工业大学,2016.\\
	$\left[2\right]$朱江辉,王富生,王安强.大展弦比复合材料机翼静气动弹性参数分析[J].机械设计与制造,2011(2):186-188.\\
	$\left[3\right]$戴凯. 基于FLUENT的飞行器气动特性仿真研究[D].西安工业大学,2015.\\
	$\left[4\right]$黄季墀,汪海.飞机结构设计与强度计算[M].上海:上海交通大学出版社,2012:78-291.\\
	$\left[5\right]$陈佳,袁朝辉,鹿思嘉.某型飞机机翼弯曲变形的仿真计算[J].机电工程,2013,30(04):422-425.\\
	$\left[6\right]$王伟,王华,贾清萍.充气机翼承载能力和气动特性分析[J].航空动力学报,2010,25(10):2296-2301.DOI:10.13224/j.cnki.jasp.2010.10.022.\\
	$\left[7\right]$孙杨,昌敏,白俊强.变形机翼飞行器发展综述[J].无人系统技术,2021,4(03)\\
	$\left[8\right]$曾攀,石亦平.工程中数值分析的复杂力学建模与高精度方法[J].中国科学基金,2000(02):24-29.DOI:10.16262/j.cnki.1000-8217.2000.02.008.\\
	$\left[9\right]$陈刚,王校培,宋军,唐军军,沈浩杰.某高载荷大后掠无人机复合材料机翼结构设计与试验验证[J].南京航空航天大学学报,2021,53(04):613-619.DOI:10.16356/j.1005-2615.2021.04.016.
\end{document}	